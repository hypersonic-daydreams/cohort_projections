% Figure Captions for Journal Article
% Forecasting International Migration to North Dakota
% Generated: 2025-12-29

\begin{figure}[htbp]
\centering
\includegraphics[width=\textwidth]{figures/fig_01_timeseries.pdf}
\caption{International migration to North Dakota, 2010--2024. Panel~(A) shows annual migration flows with a quadratic trend line. Panel~(B) displays North Dakota's share of total U.S.\ international migration, with the solid line showing observed values and the dashed line indicating the HP filter trend component ($\lambda = 6.25$). Vertical dotted lines mark identified structural breaks at 2020 (COVID-19 pandemic) and 2021 (recovery period), both significant at $p < 0.01$ using Chow tests.}
\label{fig:timeseries}
\end{figure}

\begin{figure}[htbp]
\centering
\includegraphics[width=\textwidth]{figures/fig_02_concentration.pdf}
\caption{Geographic concentration of foreign-born population in North Dakota by country of origin (2023). Bars show location quotients (LQ), where LQ $> 1$ indicates overrepresentation relative to the national average. The dashed vertical line marks LQ $= 1$. Countries are sorted by LQ value; bar intensity reflects population size (darker indicates larger foreign-born population in North Dakota). Egypt, India, and several African origin countries show the highest concentration relative to their national presence.}
\label{fig:concentration}
\end{figure}

\begin{figure}[htbp]
\centering
\includegraphics[width=0.9\textwidth]{figures/fig_03_acf_pacf.pdf}
\caption{Autocorrelation function (ACF) and partial autocorrelation function (PACF) for North Dakota international migration series. Panels~(A) and (B) show the level series, which exhibits slow decay in the ACF consistent with nonstationarity. Panels~(C) and (D) show the first-differenced series, where autocorrelations fall within the 95\% confidence bounds (dashed lines), supporting the characterization of the series as I(1). Sample size $n = 15$.}
\label{fig:acfpacf}
\end{figure}

\begin{figure}[htbp]
\centering
\includegraphics[width=\textwidth]{figures/fig_04_structural_breaks.pdf}
\caption{Structural break analysis for North Dakota international migration. Panel~(A) shows the time series with segmented regression fits before and after 2020, with Chow test statistics indicating a significant break ($F = 16.01$, $p < 0.001$). Panel~(B) displays the CUSUM test for parameter stability, where the test statistic (solid line) remains within the 95\% critical bounds (dashed lines), suggesting overall parameter stability despite the level shift.}
\label{fig:breaks}
\end{figure}

\begin{figure}[htbp]
\centering
\includegraphics[width=\textwidth]{figures/fig_05_gravity.pdf}
\caption{Gravity model estimation results. Panel~(A) shows coefficient estimates with 95\% confidence intervals for the full gravity specification: network effects (log diaspora stock), origin mass (log origin country population in U.S.), and destination mass (log state immigrant population). All coefficients are statistically significant at $p < 0.001$. Panel~(B) compares network elasticity estimates across model specifications, showing that controlling for mass variables reduces the diaspora effect from 0.36 to approximately 0.10.}
\label{fig:gravity}
\end{figure}

\begin{figure}[htbp]
\centering
\includegraphics[width=\textwidth]{figures/fig_06_event_study.pdf}
\caption{Event study estimates for the Travel Ban effect on refugee arrivals. Coefficients represent the difference in log arrivals between treated countries (Iran, Iraq, Libya, Somalia, Sudan, Syria, Yemen) and control countries relative to the reference period ($t = -1$, 2017). Blue circles show pre-treatment estimates; red squares show post-treatment effects. The pre-trend test fails to reject parallel trends ($F = 1.38$, $p = 0.149$). The average treatment effect on the treated (ATT) is $-1.38$ ($p = 0.004$), corresponding to a 74.9\% reduction in arrivals.}
\label{fig:eventstudy}
\end{figure}

\begin{figure}[htbp]
\centering
\includegraphics[width=\textwidth]{figures/fig_07_survival.pdf}
\caption{Kaplan--Meier survival curves for immigration wave duration by initial intensity quartile. Waves are defined as periods where arrivals exceed 50\% above baseline for at least two consecutive years. Q1 (lowest intensity) waves have median duration of 2 years, while Q4 (highest intensity) waves persist for a median of 4 years. The log-rank test strongly rejects equality across groups ($\chi^2 = 278.7$, $p < 0.001$), indicating that higher-intensity immigration flows are significantly more persistent.}
\label{fig:survival}
\end{figure}

\begin{figure}[htbp]
\centering
\includegraphics[width=\textwidth]{figures/fig_08_scenarios.pdf}
\caption{Projection scenarios for North Dakota international migration, 2025--2045. Four scenarios are shown: CBO Full (elevated policy implementation, 8\% annual growth), Moderate (dampened historical trend), Zero (counterfactual with no international migration), and Pre-2020 Trend (continuation of 2010--2019 trajectory). Shaded bands represent 50\% and 95\% confidence intervals from 1,000 Monte Carlo simulations. Historical data (2010--2024) shown with black circles. The vertical dashed line separates historical observations from projections.}
\label{fig:scenarios}
\end{figure}
