% Figure Captions for Journal Article
% Forecasting International Migration to North Dakota
% Generated: 2025-12-29

\begin{figure}[htbp]
\centering
\includegraphics[width=\textwidth]{figures/fig_01_timeseries.pdf}
\caption{International migration to North Dakota, 2010--2024. Panel~(A) shows annual net international migration (persons) with a quadratic trend line. Panel~(B) displays North Dakota's share of total U.S.\ net international migration, with the solid line showing observed values and the dashed line indicating the HP filter trend component ($\lambda = 6.25$). Vertical dotted lines mark candidate break years tested via Chow tests at 2020 (COVID-19 pandemic) and 2021 (post-pandemic period), both significant at $p < 0.01$.}
\label{fig:timeseries}
\end{figure}

\begin{figure}[htbp]
\centering
\includegraphics[width=\textwidth]{figures/fig_02_concentration.pdf}
\caption{Geographic concentration of foreign-born population in North Dakota by origin (2023). Bars show location quotients (LQ), defined as the ratio of an origin's share of North Dakota's foreign-born population to its share of the U.S.\ foreign-born population: $\mathrm{LQ}_i = (FB_{i,\mathrm{ND}}/FB_{\mathrm{ND}})\,/\,(FB_{i,\mathrm{US}}/FB_{\mathrm{US}})$. Values $> 1$ indicate overrepresentation relative to the national average; the dashed vertical line marks LQ $= 1$. Origins are sorted by LQ value; bar intensity reflects population size (darker indicates larger foreign-born population in North Dakota). Numbers at bar ends show the origin-specific foreign-born population in North Dakota (persons). Liberia, Ivory Coast, Somalia, and Tanzania show the highest concentration relative to their national presence.}
\label{fig:concentration}
\end{figure}

\begin{figure}[htbp]
\centering
\includegraphics[width=0.9\textwidth]{figures/fig_03_acf_pacf.pdf}
\caption{Autocorrelation function (ACF) and partial autocorrelation function (PACF) for North Dakota international migration series. Panels~(A) and (B) show the level series, which exhibits slow decay in the ACF consistent with nonstationarity. Panels~(C) and (D) show the first-differenced series, where autocorrelations fall within the 95\% confidence bounds (dashed lines), supporting the characterization of the series as I(1). Sample size $n = 15$.}
\label{fig:acfpacf}
\end{figure}

\begin{figure}[htbp]
\centering
\includegraphics[width=\textwidth]{figures/fig_04_structural_breaks.pdf}
\caption{Structural break diagnostics for North Dakota international migration. Panel~(A) shows annual net international migration with separate linear fits for 2010--2019 and 2020--2024; the vertical dotted line marks the candidate break year (2020). The Chow test rejects equality of pre/post trend parameters at 2020 ($F = 16.01$, $p = 0.0006$). Panel~(B) displays the CUSUM test for parameter stability; the statistic remains within the 95\% bounds (dashed lines), so we do not reject stability at $\alpha = 0.05$. Because the series contains only 15 annual observations, these diagnostics have limited power and should be interpreted cautiously.}
\label{fig:breaks}
\end{figure}

\begin{figure}[htbp]
\centering
\includegraphics[width=\textwidth]{figures/fig_05_gravity.pdf}
\caption{Gravity model estimation results. Panel~(A) shows coefficient estimates with 95\% confidence intervals for the full gravity specification: diaspora association (log diaspora stock), origin mass (log origin stock in U.S.), and destination mass (log state foreign-born total). Panel~(B) compares diaspora association estimates across model specifications, showing that controlling for mass variables reduces the diaspora coefficient from 0.45 to approximately 0.14.}
\label{fig:gravity}
\end{figure}

\begin{figure}[htbp]
\centering
\includegraphics[width=\textwidth]{figures/fig_06_event_study.pdf}
\caption{Event study estimates for treated--control divergence around the Travel Ban in refugee arrivals. Coefficients represent the difference in log arrivals between treated countries (Iran, Iraq, Libya, Somalia, Sudan, Syria, Yemen) and control countries relative to the reference period ($t = -1$, 2017). Blue circles show pre-treatment estimates; red squares show post-treatment estimates. The joint pre-trend test rejects parallel trends over the full pre-period ($F = 4.22$, $p < 0.001$). The average treatment effect on the treated (ATT) is $-1.39$ (conventional $p = 0.031$), corresponding to a 75.2\% lower level of arrivals for treated relative to control in the post period; because parallel trends are not supported in the full pre-period, estimates are interpreted as policy-associated divergence rather than definitive causal effects.}
\label{fig:eventstudy}
\end{figure}

\begin{figure}[htbp]
\centering
\includegraphics[width=\textwidth]{figures/fig_07_survival.pdf}
\caption{Kaplan--Meier survival curves for immigration wave duration by initial intensity quartile. Waves are defined as periods where arrivals exceed 50\% above baseline for at least two consecutive years. Q1 (lowest intensity) waves have median duration of 2 years, while Q4 (highest intensity) waves persist for a median of 6 years. The log-rank test strongly rejects equality across groups ($\chi^2 = 633.0$, $p < 10^{-136}$), indicating that higher-intensity immigration flows are significantly more persistent.}
\label{fig:survival}
\end{figure}

\begin{figure}[htbp]
\centering
\includegraphics[width=\textwidth]{figures/fig_08_scenarios.pdf}
\caption{Projection scenarios for North Dakota international migration, 2025--2045. The figure shows four non-zero scenario paths: CBO (Jan 2026) Baseline (CBO national net immigration projection scaled by North Dakota's mean share of U.S.\ net international migration in the \PEP series), Moderate (dampened historical trend), Restrictive policy (0.65$\times$ Moderate), and Pre-2020 Trend (anchored to 2019 with the 2010--2019 slope). The Zero counterfactual (no international migration) is a flat line at $y=0$ and is omitted from the plot for readability but remains part of the scenario set reported in Table~\ref{tab:scenarios}. Shaded bands show baseline-only pointwise prediction intervals from Monte Carlo simulations: the dark band is the central 50\% interval (25th--75th percentiles) and the light band is the central 95\% interval (5th--95th percentiles). The solid gray line is the baseline median. Dashed lines show a wave-adjusted 95\% envelope that adds hazard-based wave persistence draws; this envelope is conservative because wave-driven variation may already be reflected in baseline forecast uncertainty. Because some scenarios are intentionally anchored away from the final historical year (e.g., external CBO scaling, pre-2020 counterfactual anchoring, or immediate policy multipliers), paths may be discontinuous at 2025; dotted connectors indicate the change from the 2024 observation to each scenario's 2025 starting value. Pointwise intervals should not be interpreted as guaranteeing that 95\% of full simulated trajectories lie entirely within the band across all horizons. Historical data (2010--2024) shown with black circles. The vertical dashed line separates historical data from projections.}
\label{fig:scenarios}
\end{figure}
