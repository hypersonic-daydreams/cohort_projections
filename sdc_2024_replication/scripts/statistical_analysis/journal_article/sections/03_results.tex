% ==============================================================================
% RESULTS
% Forecasting International Migration to North Dakota: A Multi-Method Analysis
% ==============================================================================

\section{Results}
\label{sec:results}

This section presents findings from the nine-module analytical pipeline, organized to parallel the methodological sequence described in Section~\ref{sec:data_methods}. Results progress from descriptive characterization through time series analysis, regression-based modeling, causal inference, duration analysis, and scenario projections. Throughout, we report point estimates, standard errors, and $p$-values to facilitate comparison with prior work; however, readers should interpret significance claims cautiously given the small samples and few clusters characteristic of state-level migration analysis. Where sample sizes permit only limited inference, we emphasize effect sizes and confidence intervals over binary significance thresholds.

% ==============================================================================
% 3.1 DESCRIPTIVE PATTERNS
% ==============================================================================

\subsection{Descriptive Patterns}
\label{subsec:descriptive_patterns}

Table~\ref{tab:summary_stats} presents summary statistics for \PEP net international migration (the \PEP international migration component of change) to North Dakota and the United States over the 2010--2024 observation period. North Dakota's net international migration averaged 1,796 persons annually (SE = 383), with substantial volatility: the coefficient of variation (CV) reached 82.5\%, indicating that the standard deviation approaches the mean in magnitude. By comparison, national net international migration exhibited somewhat lower volatility (CV = 73.5\%), suggesting that state-level flows experience amplified fluctuation relative to aggregate national trends.

\begin{table}[htbp]
\centering
\caption{Summary Statistics: Net International Migration (\PEP), 2010--2024}
\label{tab:summary_stats}
\begin{tabular}{@{}lccccc@{}}
\toprule
\textbf{Variable} & \textbf{Mean} & \textbf{SD} & \textbf{CV} & \textbf{Min} & \textbf{Max} \\
\midrule
ND Net International Migration (\PEP) & 1,796 & 1,482 & 82.5\% & 30 & 5,126 \\
US Net International Migration (\PEP) & 1,010,744 & 743,049 & 73.5\% & 19,885 & 2,786,119 \\
ND Share of US Net Int'l Mig. (\%) & 0.173 & 0.054 & 31.3\% & 0.102 & 0.303 \\
ND Share of US Pop. (\%) & 0.231 & 0.006 & 2.4\% & 0.218 & 0.236 \\
\bottomrule
\end{tabular}
\begin{tablenotes}
\small
\item \textit{Notes}: $n = 15$ annual observations. SD = standard deviation; CV = coefficient of variation (SD/Mean). Net international migration data from Census Bureau \PEP (international migration component of change) vintage 2024.
\end{tablenotes}
\end{table}

The distribution of annual international migration to North Dakota exhibits positive skewness (1.10), reflecting a right tail associated with recent high-migration years. The Shapiro-Wilk test marginally fails to reject normality ($W = 0.886$, $p = 0.058$), though this finding warrants cautious interpretation given the small sample size. No observations qualify as extreme outliers under the interquartile range criterion, though the 2020 value of 30 migrants represents a dramatic departure from typical levels.

Hodrick-Prescott filter decomposition reveals a U-shaped trend in North Dakota's share of national international migration. The trend component declined from 0.211 in 2010 to a minimum of 0.157 in 2014, subsequently rising to 0.185 by 2024. This pattern suggests that North Dakota's relative attractiveness to international migrants has recovered following the mid-decade trough, potentially reflecting post-oil-boom economic restructuring and the resumption of refugee resettlement programs.

% ==============================================================================
% 3.2 GEOGRAPHIC CONCENTRATION
% ==============================================================================

\subsection{Geographic Concentration}
\label{subsec:concentration}

Analysis of geographic concentration reveals a divergence between country-level and regional-level patterns. The Herfindahl-Hirschman Index (HHI) for country-of-origin composition is 1,162 based on FY 2023 LPR admissions, classifying North Dakota's immigrant origins as ``unconcentrated'' under standard thresholds ($\text{HHI} < 1,500$). This result indicates substantial diversity at the country level, with no single origin dominating the flow.

However, regional aggregation yields markedly different conclusions. The regional-level HHI reaches 3,712, placing North Dakota firmly in the ``highly concentrated'' category ($\text{HHI} > 2,500$). This concentration derives from the dominance of two source regions: Asia accounts for 52.5\% of LPR admissions and Africa accounts for 27.9\%, together comprising over 80\% of legal permanent resident flows. North America (12.3\%), Europe (3.4\%), South America (3.4\%), and Oceania (0.6\%) contribute the remainder.

Location Quotient (LQ) analysis identifies countries substantially overrepresented in North Dakota relative to national immigrant composition. Table~\ref{tab:location_quotients} reports LQ values for the ten most overrepresented origins. Liberia exhibits the highest LQ (40.83), indicating that North Dakota's share of Liberian-born residents exceeds the national share by a factor of forty. Other highly overrepresented origins include Ivory Coast (LQ = 26.19), Somalia (LQ = 22.42), and Tanzania (LQ = 15.13). This pattern reflects the legacy of refugee resettlement programs that have directed migrants from specific African origin countries to North Dakota.

\begin{table}[htbp]
\centering
\caption{Location Quotients: Top Overrepresented Origin Countries, 2023}
\label{tab:location_quotients}
\begin{tabular}{@{}lrrrr@{}}
\toprule
\textbf{Country} & \textbf{ND Foreign-Born} & \textbf{ND Share (\%)} & \textbf{US Share (\%)} & \textbf{LQ} \\
\midrule
Liberia & 3,151 & 9.05 & 0.22 & 40.83 \\
Ivory Coast & 627 & 1.80 & 0.07 & 26.19 \\
Somalia & 1,631 & 4.69 & 0.21 & 22.42 \\
Tanzania & 373 & 1.07 & 0.07 & 15.13 \\
DR Congo & 610 & 1.75 & 0.13 & 13.29 \\
Bhutan & 329 & 0.95 & 0.10 & 9.86 \\
Sudan & 400 & 1.15 & 0.12 & 9.23 \\
Zimbabwe & 164 & 0.47 & 0.06 & 8.21 \\
Kenya & 903 & 2.59 & 0.36 & 7.26 \\
Nepal & 860 & 2.47 & 0.37 & 6.63 \\
\bottomrule
\end{tabular}
\begin{tablenotes}
\small
\item \textit{Notes}: LQ = Location Quotient. Values exceeding 1.0 indicate overrepresentation relative to national composition. Data from American Community Survey 2023.
\end{tablenotes}
\end{table}

The concentration patterns exhibit meaningful temporal evolution. Country-level HHI has increased from 482 in 2012 to 942 in 2023, though the trend is not statistically significant ($p = 0.242$). Regional-level HHI shows more pronounced increases, rising from approximately 2,600 in the early 2010s to over 5,200 in 2022--2023, reflecting growing reliance on African and Asian source regions.

% ==============================================================================
% 3.3 TIME SERIES PROPERTIES
% ==============================================================================

\subsection{Time Series Properties}
\label{subsec:time_series_properties}

Unit root diagnostics inform integration choices for subsequent time series modeling. Table~\ref{tab:unit_root} reports Augmented Dickey-Fuller (ADF) test results for the three primary series.

\begin{table}[htbp]
\centering
\caption{Unit Root Test Results}
\label{tab:unit_root}
\begin{tabular}{@{}lcccc@{}}
\toprule
\textbf{Variable} & \textbf{ADF (Level)} & \textbf{$p$-value} & \textbf{ADF (Diff)} & \textbf{$p$-value} \\
\midrule
ND International Migration & $-1.453$ & 0.556 & $-3.843$ & 0.002 \\
ND Share of US & $-2.004$ & 0.285 & $-6.214$ & $<0.001$ \\
US International Migration & $-3.908$ & 0.002 & $-2.347$ & 0.157 \\
\midrule
\textbf{Integration Order} & & & & \\
ND International Migration & \multicolumn{4}{c}{I(1) --- First-difference stationary} \\
ND Share of US & \multicolumn{4}{c}{I(1) --- First-difference stationary} \\
US International Migration & \multicolumn{4}{c}{I(0) --- Level stationary} \\
\bottomrule
\end{tabular}
\begin{tablenotes}
\small
\item \textit{Notes}: ADF = Augmented Dickey-Fuller test statistic. Lag selection by AIC. $n = 15$ annual observations. Critical values at 5\%: $-3.29$ (level), $-3.13$ (differenced).
\end{tablenotes}
\end{table}

North Dakota international migration does not reject the unit root null in levels ($p = 0.556$) but rejects after first differencing ($p = 0.002$), consistent with I(1) behavior. North Dakota's share of national migration shows the same ADF pattern (level $p = 0.285$; differenced $p < 0.001$). In contrast, national international migration rejects the unit root null in levels ($p = 0.002$), consistent with I(0) behavior.

KPSS tests (null: level-stationarity; regression = constant) do not reject level-stationarity for the ND migration series (p $\ge 0.10$), while KPSS rejects level-stationarity for the first-differenced ND share series (p $\approx 0.042$). Together with ADF results, the small-sample diagnostics are suggestive rather than definitive: integration order is treated as a modeling choice, with break-robust checks (Appendix~\ref{app:unit_root_robustness}) used to evaluate sensitivity.\footnote{Statsmodels reports KPSS p-values as 0.10 when p $\ge 0.10$; values shown as 0.10 should be read as upper bounds.}

ARIMA model selection via AIC identifies ARIMA(0,1,0) as a candidate baseline specification in this very small sample. The random-walk baseline implies that the best one-step forecast equals the most recent observation, with innovations representing unpredictable shocks. Diagnostic tests support adequacy within the limited sample: the Ljung-Box portmanteau test finds no significant residual autocorrelation through lag 5 ($Q = 4.12$, $p = 0.404$), and residuals pass the Shapiro-Wilk normality test ($W = 0.968$, $p = 0.820$).

The random-walk baseline implies substantial forecast uncertainty. Five-year-ahead prediction intervals widen dramatically: the 2029 forecast of 5,126 migrants carries a 95\% prediction interval of [212, 10,040], spanning nearly two orders of magnitude. This uncertainty reflects both the inherent unpredictability of migration processes and the limited sample size constraining parameter precision.

Rolling-origin backtesting (expanding window, 2010--2016 start; 2017--2024 forecasts) benchmarks point accuracy and interval calibration. Table~\ref{tab:backtesting} reports Mean Absolute Scaled Error (MASE) alongside conventional metrics; MASE is robust to the near-zero 2020 observation that inflates MAPE. The driver regression using contemporaneous U.S.\ migration---an ``oracle'' benchmark that assumes knowledge of future national flows---achieves the lowest MAE/RMSE and MASE = 0.27. However, this model is infeasible for real-time forecasting. A feasible variant using lagged (prior-year) U.S.\ migration achieves MASE = 1.06, essentially matching the naive baseline. Among all feasible methods, the naive random-walk and ARIMA(0,1,0) perform identically (MASE = 1.00), with no feasible model significantly outperforming the random walk. Interval widths remain large across models; these results are therefore descriptive rather than definitive.

\begin{table}[htbp]
\centering
\caption{Rolling-Origin Backtesting Results (h = 1)}
\label{tab:backtesting}
\begin{tabular}{@{}lrrrrrr@{}}
\toprule
\textbf{Model} & \textbf{MAE} & \textbf{RMSE} & \textbf{MASE} & \textbf{MAPE} & \textbf{Cov. 80\%} & \textbf{Cov. 95\%} \\
\midrule
\multicolumn{7}{l}{\textit{Feasible methods (use only past data):}} \\
Naive RW & 1,153 & 1,367 & 1.00 & 313\% & 0.63 & 0.75 \\
Expanding mean & 1,674 & 1,980 & 1.45 & 635\% & 0.38 & 0.50 \\
Driver OLS (Lagged) & 1,224 & 1,408 & 1.06 & 534\% & 0.38 & 0.63 \\
ARIMA(0,1,0) & 1,153 & 1,367 & 1.00 & 313\% & 0.63 & 0.75 \\
\midrule
\multicolumn{7}{l}{\textit{Infeasible benchmark (requires future data):}} \\
Driver OLS (Oracle)$^\dagger$ & 317 & 530 & 0.27 & 65\% & 0.88 & 0.88 \\
\bottomrule
\end{tabular}
\begin{tablenotes}
\small
\item \textit{Notes}: Rolling-origin evaluation with expanding window; $n = 8$ one-step forecasts. MASE = Mean Absolute Scaled Error (scaled by naive RW MAE; values $<$1 outperform naive baseline). Coverage refers to empirical coverage of symmetric prediction intervals. MAPE is inflated by the near-zero 2020 observation; MASE provides a more robust accuracy comparison.
\item[$\dagger$] \textit{Oracle benchmark}: Driver OLS with contemporaneous U.S.\ migration values, unavailable at forecast time. The lagged driver---a feasible variant using prior-year U.S.\ migration---achieves MASE = 1.06, essentially matching the naive baseline. No feasible model significantly outperforms random walk.
\end{tablenotes}
\end{table}

Autocorrelation function (ACF) and partial autocorrelation function (PACF) plots for the differenced series (Figure~\ref{fig:acfpacf}) confirm the absence of significant serial correlation at standard lags, consistent with the random-walk baseline.

% ==============================================================================
% 3.4 STRUCTURAL BREAK ANALYSIS
% ==============================================================================

\subsection{Structural Break Analysis}
\label{subsec:structural_breaks}

Structural break tests evaluate whether regression parameters remain stable across the observation period or exhibit discrete shifts at specific dates. Table~\ref{tab:structural_breaks} reports Chow test results for three candidate break points corresponding to major policy events: the 2017 Travel Ban, the 2020 COVID-19 pandemic, and the 2021 post-pandemic recovery.

\begin{table}[htbp]
\centering
\caption{Structural Break Test Results (Chow Tests)}
\label{tab:structural_breaks}
\begin{tabular}{@{}lcccc@{}}
\toprule
\textbf{Break Year} & \textbf{Policy Event} & \textbf{$F$-statistic} & \textbf{$p$-value} & \textbf{Regime Shift} \\
\midrule
2017 & Travel Ban & 1.29 & 0.314 & Not significant \\
2020 & COVID-19 & 16.01 & $<0.001$ & $+91.1\%$ \\
2021 & Post-COVID Recovery & 10.28 & 0.003 & $+161.6\%$ \\
\bottomrule
\end{tabular}
\begin{tablenotes}
\small
\item \textit{Notes}: Chow test with df = (2, 11). Critical values: 3.98 (5\%), 7.21 (1\%). Regime shift indicates percentage change in mean migration between pre- and post-break periods.
\end{tablenotes}
\end{table}

The 2020 COVID-19 pandemic represents a highly significant structural break ($F = 16.01$, $p < 0.001$). Mean international migration to North Dakota increased from 1,378 in the pre-2020 period to 2,633 in the post-2020 period---a 91.1\% increase. However, this result requires careful interpretation: the 2020 observation itself (30 migrants) represents partial-year data during severe travel restrictions, while subsequent years (2021--2024) reflect both recovery and potentially accelerated growth. The break thus captures a shift to a higher-volatility regime rather than a simple level increase.

The 2021 break point similarly proves significant ($F = 10.28$, $p = 0.003$), with mean migration increasing from 1,255 (2010--2020) to 3,284 (2021--2024)---a 161.6\% change. This breakpoint more cleanly separates the COVID-affected period from the subsequent recovery.

Notably, the 2017 Travel Ban shows no statistically significant break at the aggregate North Dakota level ($F = 1.29$, $p = 0.314$). This null result contrasts with the significant Travel Ban effects identified in the nationality-level difference-in-differences analysis (Section~\ref{subsec:causal_did}), suggesting that aggregate flows may mask compositional shifts in origin countries.

The CUSUM test for parameter stability corroborates these findings. The cumulative sum of recursive residuals remains within the 5\% significance bounds throughout the observation period (maximum CUSUM = 2.37, bound = 10.25), indicating no evidence of continuous parameter drift. The stability ratio of 0.23 suggests that cumulative deviations from the fitted model remain modest relative to permissible fluctuation.

The Bai-Perron procedure for endogenous break detection identifies zero breaks under the BIC criterion, reflecting the penalty imposed on additional parameters in the short series. This result highlights the tension between data-driven and theory-driven break identification in small samples.

% ==============================================================================
% 3.5 PANEL DATA RESULTS
% ==============================================================================

\subsection{Panel Data Results}
\label{subsec:panel_results}

Panel analysis exploits cross-state and temporal variation to examine patterns of international migration allocation. The estimation sample comprises 765 state-year observations (51 states $\times$ 15 years), forming a balanced panel.

The Hausman specification test compares fixed effects and random effects estimators. Under the null hypothesis that random effects estimates are consistent and efficient, the test statistic follows a chi-squared distribution. The computed test statistic is effectively zero ($H = 7.18 \times 10^{-31}$, $p = 1.000$), indicating no systematic difference between estimators and thus supporting random effects as the preferred specification. The Breusch-Pagan LM test strongly rejects pooled OLS in favor of the random effects model ($\chi^2 = 1,502$, $p < 0.001$), confirming the presence of state-level heterogeneity requiring panel methods.

State fixed effects from the two-way specification reveal substantial cross-sectional variation in international migration levels. Florida exhibits the largest positive effect ($+$125,526 relative to the grand mean), followed by California ($+$106,259) and Texas ($+$89,232). North Dakota's fixed effect of $-$18,022 places it near the bottom of the distribution, exceeded in negativity only by Wyoming ($-$19,332). This ranking accords with population-weighted expectations and confirms North Dakota's status as a minor destination in absolute terms.

Year fixed effects capture common temporal shocks affecting all states. The 2020 effect is dramatically negative, reflecting the near-cessation of international migration during pandemic travel restrictions. Mean international migration in 2020 was 390 compared to 21,206 in other years---a 98.2\% reduction. The between-within variance ratio of 1.30 indicates that cross-state variation slightly exceeds within-state temporal variation, though both dimensions contribute meaningfully to total variability.

% ==============================================================================
% 3.6 CROSS-SECTIONAL ALLOCATION MODEL ESTIMATES
% ==============================================================================

\subsection{Cross-Sectional Allocation Model Estimates}
\label{subsec:gravity_results}

Cross-sectional allocation models estimated via Poisson pseudo-maximum likelihood (PPML) quantify the relationship between LPR admissions and diaspora stocks. As noted in Section~\ref{subsec:gravity_methods}, these models differ from traditional gravity specifications in that they omit origin--destination distance, which varies minimally across U.S.\ states for a given origin country. Table~\ref{tab:gravity} reports coefficient estimates across three specifications of increasing complexity.

\begin{table}[htbp]
\centering
\caption{Cross-Sectional Allocation Model Estimates (PPML)}
\label{tab:gravity}
\begin{tabular}{@{}lccc@{}}
\toprule
\textbf{Variable} & \textbf{Simple} & \textbf{Full Model} & \textbf{State FE} \\
\midrule
Log(Diaspora Stock) & 0.448*** & 0.140 & 0.163 \\
 & (0.104) & (0.248) & (0.148) \\
Log(Origin Stock in U.S.) & --- & 0.046 & --- \\
 & & (0.192) & \\
Log(State Foreign-Born Total) & --- & 0.792*** & --- \\
 & & (0.212) & \\
Constant & 1.267 & $-7.711$* & --- \\
 & (0.980) & (3.123) & \\
\midrule
$n$ & 4,845 & 4,845 & 4,845 \\
Pseudo $R^2$ & 0.236 & 0.399 & 0.413 \\
\bottomrule
\end{tabular}
\begin{tablenotes}
\small
\item \textit{Notes}: Two-way clustered standard errors (state and origin) in parentheses. PPML estimation. * $p < 0.05$; *** $p < 0.001$. Dependent variable: LPR admissions by state-country pair, FY 2023. Distance is omitted because within-U.S.\ variation in origin--state distance is minimal (see Section~\ref{subsec:gravity_methods}). State FE specification includes 50 state dummy variables (coefficients not reported).
\end{tablenotes}
\end{table}

The simple specification yields a diaspora association of 0.448 (SE = 0.104, $p < 0.001$), indicating that larger state--origin stocks are correlated with higher FY2023 admissions. However, this bivariate model omits origin and destination scale.

The full model adds origin and destination foreign-born totals. With two-way clustered SEs, the diaspora association attenuates to 0.140 (SE = 0.248, $p = 0.57$), while state foreign-born totals remain the dominant predictor (elasticity = 0.792, SE = 0.212, $p < 0.001$). The origin stock coefficient is small and imprecise (0.046, SE = 0.192).

The state fixed effects specification yields a diaspora association of 0.163 (SE = 0.148, $p = 0.27$). This estimate remains below the simple model and underscores the sensitivity of cross-sectional coefficients to controls and inference.

The attenuation from 0.45 to approximately 0.14 has practical implications for forecasting. Much of the simple-model association reflects selection of large receiving states with both large diaspora stocks and continued large flows. The pseudo-$R^2$ values indicate that the full model explains 40\% of deviance, roughly double the explanatory power of the bivariate specification.

Because FY2023 is a single cross-section, these results are interpreted as predictive diaspora--flow associations rather than causal diaspora effects. We report two-way clustered SEs to account for correlation within origins and destinations, and note that ACS diaspora stocks are estimates with margins of error; coefficient uncertainty does not fully propagate that measurement error.

% ==============================================================================
% 3.7 CAUSAL INFERENCE FINDINGS
% ==============================================================================

\subsection{Causal Inference Findings}
\label{subsec:causal_did}

Causal inference results focus on difference-in-differences for the Travel Ban and an interrupted time series for COVID-19. A synthetic comparator is reported as a descriptive benchmark, and a shift-share index provides first-stage relevance for refugee-driven shocks.

\begin{table}[htbp]
\centering
\caption{Policy Effects and Predictive Relevance}
\label{tab:causal_effects}
\begin{tabular}{@{}llccc@{}}
\toprule
\textbf{Method} & \textbf{Policy} & \textbf{Estimate} & \textbf{SE} & \textbf{95\% CI} \\
\midrule
DiD (log scale) & Travel Ban & $-1.384$* & 0.646 & [$-2.65$, $-0.12$] \\
\quad \textit{Percentage effect} & & $-74.9\%$ & --- & --- \\
ITS Level Shift & COVID-19 & $-19,503$*** & 4,556 & [$-28,432$, $-10,573$] \\
ITS Trend Change & COVID-19 & $+14,113$*** & 3,288 & [$+7,667$, $+20,558$] \\
\midrule
Shift-share (first stage) & National shocks & 5.51*** & 0.87 & [$+3.80$, $+7.22$] \\
\quad First-stage $F$ & & 39.95 & & \\
\bottomrule
\end{tabular}
\begin{tablenotes}
\small
\item \textit{Notes}: DiD = Difference-in-differences with nationality and year fixed effects, SEs clustered by nationality. ITS = Interrupted time series with state fixed effects, SEs clustered by state. Shift-share row reports first-stage relevance only (not a causal effect). * $p < 0.05$; *** $p < 0.001$. Travel Ban affected countries: Iran, Iraq, Libya, Somalia, Sudan, Syria, Yemen.
\end{tablenotes}
\end{table}

\subsubsection{Travel Ban Difference-in-Differences}

The Travel Ban analysis employs refugee arrival data spanning 2002--2019, with 2018 designated as the first full post-treatment year. The treatment group comprises seven affected nationalities (Iran, Iraq, Libya, Somalia, Sudan, Syria, Yemen), while 119 unaffected nationalities serve as controls. The sample includes 1,137 nationality-year observations.

The estimated average treatment effect on the treated (ATT) is $-1.384$ on the log scale ($t = -2.14$, $p = 0.032$), corresponding to a policy-associated divergence of 74.9\% in refugee arrivals from affected countries. The 95\% confidence interval [$-2.65$, $-0.12$] excludes zero, indicating a statistically significant and economically substantial short-run divergence.

Parallel trends diagnostics are mixed, requiring caution in causal interpretation. The linear pre-trend test yields a treated-country trend coefficient of 0.087 ($t = 0.97$, $p = 0.334$), but the event study's joint pre-treatment test rejects parallel trends over the full pre-period ($F = 4.31$, $p < 0.001$). Figure~\ref{fig:eventstudy} shows divergence beginning approximately 10 years before treatment, indicating that the treatment and control groups were on different trajectories prior to the policy. The immediate post-treatment coefficients shift sharply negative in 2018--2019, but because the parallel trends assumption is violated, the estimated effect conflates the policy impact with pre-existing divergence. We therefore interpret the 75\% estimate as a conservative lower bound on the causal policy effect rather than a point estimate of the true effect.

\textbf{Small-sample inference robustness.} With only seven treated clusters (nationalities), conventional cluster-robust inference may be unreliable. Table~\ref{tab:bootstrap_inference} reports alternative $p$-values from wild cluster bootstrap and randomization inference. For the full pre-period specification, the wild cluster bootstrap yields $p = 0.077$ and randomization inference yields $p = 0.058$---both above the conventional $\alpha = 0.05$ threshold, though significant at $\alpha = 0.10$. This attenuation relative to the conventional $p = 0.032$ reflects the small number of treated clusters and suggests caution in claiming strong statistical significance.

\begin{table}[htbp]
\centering
\caption{Small-Sample Inference Robustness for Travel Ban DiD}
\label{tab:bootstrap_inference}
\begin{tabular}{@{}lcccc@{}}
\toprule
\textbf{Specification} & \textbf{ATT} & \textbf{Conventional $p$} & \textbf{WCB $p$} & \textbf{RI $p$} \\
\midrule
Full pre-period (2002--2017) & $-1.384$ & 0.032* & 0.077 & 0.058 \\
Restricted (2013--2017) & $-1.983$ & $<$0.001*** & 0.003** & 0.004** \\
\bottomrule
\end{tabular}
\begin{tablenotes}
\small
\item \textit{Notes}: WCB = Wild cluster bootstrap with Rademacher weights (999 iterations). RI = Randomization inference with treatment permutation (999 permutations). * $p < 0.05$; ** $p < 0.01$; *** $p < 0.001$.
\end{tablenotes}
\end{table}

\textbf{Restricted pre-period robustness.} Because the event study reveals greater pre-trend divergence in early years (2002--2012) than in years immediately preceding treatment, we re-estimate the DiD restricting to years where parallel trends appear more plausible. Restricting the pre-period to 2013--2017 substantially improves the pre-trend diagnostic: the joint $F$-test yields $F = 0.76$, $p = 0.553$, failing to reject parallel trends. The ATT estimate increases in magnitude to $-1.983$ (SE = 0.506), and inference strengthens: the conventional $p$-value falls to $p < 0.001$, with bootstrap ($p = 0.003$) and randomization ($p = 0.004$) inference confirming significance at the 1\% level.

This pattern---larger ATT magnitude and stronger significance with the restricted pre-period---suggests that the full-sample estimate is attenuated by pre-existing negative divergence between treatment and control groups. When analysis focuses on years where parallel trends are more defensible, the estimated policy effect is larger ($-$86\% vs.\ $-$75\%) and more precisely estimated. The restricted-period results strengthen confidence that the Travel Ban produced a genuine reduction in refugee arrivals from affected countries, though the magnitude remains subject to specification uncertainty.

\subsubsection{COVID-19 Interrupted Time Series}

The COVID-19 analysis employs state-level panel data (2010--2024) with 2020 as the intervention year. The interrupted time series model identifies both level and trend effects.

\textbf{Scope of inference:} The ITS specification includes state fixed effects and estimates the \emph{average} COVID disruption across all U.S.\ states, not a North Dakota-specific effect. Because COVID-19 and associated travel restrictions affected all states simultaneously, the model cannot separately identify whether North Dakota experienced a differential disruption relative to other states. The coefficients reported below describe the mean state-level response; ND-specific deviations from this average are absorbed by the state fixed effect and are not separately estimated. Extending the model to include ND-specific interaction terms (e.g., $\text{ND} \times \text{Post}_{2020}$) would require additional identifying assumptions and is left for future work.

The level shift coefficient indicates that international migration dropped by 19,503 persons immediately following COVID onset ($t = -4.28$, $p < 0.001$). The trend change coefficient of $+$14,113 ($t = 4.29$, $p < 0.001$) reflects the steep post-pandemic rebound slope during 2021--2024. Relative to pre-COVID mean migration of 15,669, the level shift represents a 124.5\% decline---interpretable as a transition from positive to negative net migration during the acute pandemic period.

\textbf{Interpretive caveat:} The large positive trend change coefficient is a mathematical artifact of fitting a linear trend to the short, steep recovery period (2021--2024). With pre-COVID mean migration of approximately 1,800 persons annually, a coefficient of +14,113 would imply explosive growth if extrapolated---clearly implausible. This coefficient should be interpreted as describing the short-run ``rebound slope'' rather than a sustainable long-term trajectory. Extrapolation beyond the observed post-COVID window is not warranted.

\subsubsection{Synthetic Comparator (Descriptive Benchmark)}

Because the Travel Ban is a national shock, we do not interpret a state-level synthetic comparator as causal. Instead, we construct a synthetic comparator that matches North Dakota's pre-2017 international migration rate using weighted donor states. The weights emphasize Wyoming (41.9\%), Vermont (24.5\%), Rhode Island (20.1\%), and Washington (8.3\%), with minor contributions from Oregon (2.7\%) and Florida (2.5\%).

Pre-treatment fit is strong: the root mean squared prediction error (RMSPE) of 0.020 indicates close tracking during 2010--2016. The post-2017 gap averages +0.57 migrants per 1,000 population (SD = 0.78), and the post/pre RMSPE ratio is large (48.6), indicating substantial divergence. These deviations are presented descriptively and are not interpreted as policy effects.

\subsubsection{Shift-Share (Bartik) Index}

The shift-share index combines baseline state shares with leave-one-out national origin shocks to summarize predicted refugee inflows. The first-stage $F$-statistic of 39.95 exceeds the conventional threshold of 10, supporting instrument relevance.

The estimated first-stage coefficient is 5.51 ($p < 0.001$), indicating that a one-unit increase in the shift-share index predicts a 5.51-unit increase in state-level international migration. Because no second-stage causal model is estimated, this result is interpreted as predictive relevance rather than a causal effect.

% ==============================================================================
% 3.8 DURATION ANALYSIS
% ==============================================================================

\subsection{Duration Analysis}
\label{subsec:duration_results}

Duration analysis examines the lifecycle of immigration ``waves''---sustained periods of elevated arrivals from specific origin countries. Wave identification criteria require arrivals to exceed 150\% of baseline for at least two consecutive years. This procedure identifies 940 waves across 56 nationalities and 48 states in the 2002--2020 refugee data.

Kaplan-Meier estimation yields a median wave duration of 3.0 years (mean = 3.54 years). Survival probabilities decline steadily over the wave lifecycle: 51.6\% of waves survive past year 2; 31.8\% survive past year 3; 21.6\% survive past year 4; and only 6.2\% survive past year 10. The censoring rate of 10\% reflects waves still active at the end of the observation period (FY 2020).

Stratified analysis reveals significant heterogeneity by wave intensity. Table~\ref{tab:survival} reports median survival by intensity quartile.

\begin{table}[htbp]
\centering
\caption{Wave Duration by Intensity Quartile}
\label{tab:survival}
\begin{tabular}{@{}lccc@{}}
\toprule
\textbf{Intensity Quartile} & \textbf{$n$} & \textbf{Median Survival} & \textbf{Mean Duration} \\
\midrule
Q1 (Low) & 247 & 2.0 years & 2.43 years \\
Q2 & 223 & 2.0 years & 2.80 years \\
Q3 & 236 & 3.0 years & 3.61 years \\
Q4 (High) & 234 & 4.0 years & 5.35 years \\
\midrule
\multicolumn{4}{l}{\textit{Log-rank test}: $\chi^2 = 278.7$, $p < 10^{-60}$} \\
\bottomrule
\end{tabular}
\begin{tablenotes}
\small
\item \textit{Notes}: Intensity = peak arrivals relative to baseline. Duration measured in years. Log-rank test evaluates equality of survival curves across quartiles.
\end{tablenotes}
\end{table}

The log-rank test decisively rejects equality of survival curves across intensity quartiles ($\chi^2 = 278.7$, $p < 10^{-60}$), confirming that higher-intensity waves persist substantially longer. The median survival differential between Q1 and Q4---two years versus four years---indicates that wave duration doubles at the upper extreme of initial intensity.

Regional origin also influences wave persistence. African (mean duration = 3.99 years), Middle Eastern (4.14 years), and Asian (4.28 years) waves outlast those from Europe (2.40 years) and the Americas (2.25 years). The log-rank test confirms significant regional heterogeneity ($\chi^2 = 89.7$, $p < 10^{-17}$).

Cox proportional hazards regression quantifies the multivariate relationship between wave characteristics and termination risk. Key findings include:

\begin{itemize}
    \item \textbf{Log intensity}: HR = 0.412 ($p < 0.001$). Higher-intensity waves exhibit 59\% lower hazard of termination, confirming the protective effect of initial magnitude.
    \item \textbf{Early wave}: HR = 1.361 ($p < 0.001$). Waves beginning in the early observation period terminate faster, potentially reflecting data truncation or secular trends.
    \item \textbf{Peak arrivals}: HR = 0.656 ($p < 0.001$). Higher peak arrivals during the wave reduce termination hazard.
    \item \textbf{Americas origin}: HR = 1.711 ($p = 0.005$). Waves from the Americas terminate 71\% faster than the reference category (Africa).
    \item \textbf{Europe origin}: HR = 1.568 ($p = 0.004$). European-origin waves similarly exhibit elevated termination hazard.
\end{itemize}

The Cox model achieves a concordance index of 0.769, indicating good discriminative ability---the model correctly ranks wave pairs by survival time 77\% of the time. The proportional hazards assumption is satisfied based on the global Schoenfeld residual test.

To connect duration estimates to forecasting, we translate the Cox model into year-ahead survival probabilities for any active wave. For a wave observed at age $a$, the conditional survival ratio $S(a+k \mid x) / S(a \mid x)$ yields the probability the wave persists $k$ additional years, which we combine with the empirically estimated lifecycle shape (initiation $\rightarrow$ peak $\rightarrow$ decline) to simulate wave contributions within the Monte Carlo scenario engine. Figure~\ref{fig:survival} summarizes the survival curves that anchor these persistence draws.

% ==============================================================================
% 3.9 FORECAST SCENARIOS
% ==============================================================================

\subsection{Forecast Scenarios}
\label{subsec:scenario_results}

Scenario analysis translates analytical findings into forward-looking projections through 2045. Four policy-indexed scenarios span the range of plausible trajectories, while Monte Carlo simulation quantifies parametric uncertainty. Although the Driver OLS model achieves superior backtesting accuracy (Table~\ref{tab:backtesting}), it requires contemporaneous knowledge of national migration---information unavailable at forecast time. Long-term scenarios therefore employ the conservative ARIMA baseline, which relies only on historical data and represents the best-performing feasible univariate method. Table~\ref{tab:scenarios} summarizes 2045 projections under each scenario.

\begin{table}[htbp]
\centering
\caption{Net International Migration (\PEP) Scenario Projections for North Dakota, 2045 (Units: Persons, Net)}
\label{tab:scenarios}
\begin{tabular}{@{}llc@{}}
\toprule
\textbf{Scenario} & \textbf{Assumptions} & \textbf{2045 Projection} \\
\midrule
CBO Full & 2025--2029: 1.1$\times$ARIMA; 2030--2045: 8\% growth & 19,318 \\
Moderate & Dampened historical trend (50\% weight) & 7,048 \\
Pre-2020 Trend & Anchor 2019; continue 2010--2019 slope (+72/year) & 2,517 \\
Immigration Policy & Reduced admissions via SDC factor (0.65x) & 3,893 \\
Zero & No international migration & 0 \\
\midrule
Monte Carlo Median & Stochastic trend with $CV_{2045} = 0.38$ & 9,056 \\
\quad 50\% PI & & [6,450, 11,292] \\
\quad 95\% PI & & [3,570, 14,491] \\
\bottomrule
\end{tabular}
\begin{tablenotes}
\small
\item \textit{Notes}: CBO = Congressional Budget Office elevated immigration scenario. PI = prediction interval from 1,000 Monte Carlo draws. $CV_{2045}$ denotes the coefficient of variation of simulated 2045 outcomes (historical CV is 82.5\%). Baseline 2024 value = 5,126.
\end{tablenotes}
\end{table}

The four deterministic scenarios yield dramatically different endpoints. The CBO Full scenario applies 10\% above-ARIMA levels for 2025--2029 and then compounds at 8\% annually, projecting net international migration of 19,318 persons by 2045. The Moderate scenario, applying 50\% dampening to historical trends, yields 7,048. The Pre-2020 Trend scenario anchors at the 2019 level and extrapolates the 2010--2019 slope, projecting only 2,517---reflecting the modest pre-pandemic growth rate. Finally, the Immigration Policy scenario, which applies a 35\% reduction to baseline rates (reflecting restrictive admission caps), yields 3,893 migrants---a value intermediate between the pre-pandemic trend and the moderate growth scenario.

Monte Carlo simulation propagates parameter uncertainty through the projection model. Based on 1,000 draws from estimated trend and volatility distributions ($CV_{2045} = 0.38$), the median 2045 projection is 9,056 persons, with a 50\% prediction interval of [6,450, 11,292] and a 95\% prediction interval of [3,570, 14,491]. The latter interval spans a factor of 4.1, reflecting substantial irreducible uncertainty over a 20-year horizon given the observed volatility and structural instability. The simulation CV refers to dispersion in projected 2045 outcomes and is not directly comparable to the historical CV of annual observations.

Model averaging weights alternative time series specifications by AIC. The VAR model receives effectively all AIC weight (1.000) relative to the univariate ARIMA (weight $\approx 0$), reflecting the superior fit of the multivariate specification. For cross-sectional models, Random Forest receives the highest $R^2$-based weight (0.462), followed by Elastic Net (0.293) and OLS (0.245).

\paragraph{Role of Machine Learning Methods.} The ML methods described in Section~\ref{subsec:ml_methods} serve an auxiliary role in this analysis rather than contributing directly to the scenario trajectories. Elastic Net regularization ($R^2_{CV} = 0.60$) identifies three non-zero predictors of state-level international migration share: log population (coefficient = 0.015), natural change (0.011), and birth rate ($-0.003$). Random Forest ($R^2_{CV} = 0.94$) confirms that log population dominates feature importance (93\% of relative importance), with death rate and domestic migration rate contributing marginally. K-means clustering ($K = 2$; silhouette = 0.64) places North Dakota in the larger cluster containing 50 states (including DC), with only one state forming a distinct outlier cluster.

These ML results provide three contributions to the broader analysis: (1) they validate that population scale is the primary determinant of state-level migration allocation, consistent with the cross-sectional models in Section~\ref{subsec:gravity_results}; (2) they identify peer states for North Dakota (the 51 states in its cluster) useful for comparative benchmarking; and (3) they confirm that no parsimonious feature set substantially improves upon simple population-based allocation models. Because ML methods are trained on cross-sectional variation and do not generate time-series forecasts, they do not directly parameterize the scenario engine. Full ML diagnostics are available in the supplementary materials.

The fan chart visualization (Figure~\ref{fig:scenarios}) depicts the scenario trajectories with Monte Carlo uncertainty bands, illustrating both the range of policy-contingent outcomes and the widening uncertainty envelope over the projection horizon.
