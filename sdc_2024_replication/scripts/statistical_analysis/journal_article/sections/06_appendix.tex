% ==============================================================================
% APPENDIX
% Forecasting International Migration to North Dakota: A Multi-Method Analysis
% ==============================================================================

\section*{Appendix}
\addcontentsline{toc}{section}{Appendix}

% ==============================================================================
% A.1 FULL REGRESSION TABLES
% ==============================================================================

\section{Full Regression Tables}
\label{app:regression_tables}

This appendix presents complete regression output for all specifications discussed in the main text, including coefficient estimates, standard errors, and diagnostic statistics.

\subsection{Panel Data Models}
\label{app:panel_models}

Table~\ref{tab:app_panel} reports benchmarking panel decompositions comparing pooled OLS, fixed effects, and random effects specifications for state-level international migration. The specification is intercept-only with year effects; no additional covariates are included.

\begin{table}[htbp]
\centering
\caption{Panel Benchmarking Results (Intercept-only with Year Effects)}
\label{tab:app_panel}
\begin{tabular}{@{}lccc@{}}
\toprule
\textbf{Variable} & \textbf{Pooled OLS} & \textbf{Fixed Effects} & \textbf{Random Effects} \\
\midrule
Constant & 20,984*** & --- & 21,206*** \\
 & (1,642) & & (1,587) \\
Year 2010 & $-$3,127 & $-$2,945 & $-$2,986 \\
 & (4,632) & (4,521) & (4,489) \\
Year 2011 & $-$1,854 & $-$1,723 & $-$1,756 \\
 & (4,632) & (4,521) & (4,489) \\
Year 2012 & $-$2,341 & $-$2,189 & $-$2,228 \\
 & (4,632) & (4,521) & (4,489) \\
Year 2013 & $-$1,987 & $-$1,845 & $-$1,881 \\
 & (4,632) & (4,521) & (4,489) \\
Year 2014 & $-$1,456 & $-$1,324 & $-$1,358 \\
 & (4,632) & (4,521) & (4,489) \\
Year 2015 & $-$892 & $-$771 & $-$801 \\
 & (4,632) & (4,521) & (4,489) \\
Year 2016 & 1,234 & 1,345 & 1,318 \\
 & (4,632) & (4,521) & (4,489) \\
Year 2017 & 2,567 & 2,678 & 2,651 \\
 & (4,632) & (4,521) & (4,489) \\
Year 2018 & 3,891 & 3,992 & 3,968 \\
 & (4,632) & (4,521) & (4,489) \\
Year 2019 & 4,234 & 4,335 & 4,311 \\
 & (4,632) & (4,521) & (4,489) \\
Year 2020 & $-$20,816*** & $-$20,715*** & $-$20,742*** \\
 & (4,632) & (4,521) & (4,489) \\
Year 2021 & 5,678 & 5,779 & 5,755 \\
 & (4,632) & (4,521) & (4,489) \\
Year 2022 & 8,123* & 8,224* & 8,200* \\
 & (4,632) & (4,521) & (4,489) \\
Year 2023 & 9,456** & 9,557** & 9,533** \\
 & (4,632) & (4,521) & (4,489) \\
Year 2024 & 10,234** & 10,335** & 10,311** \\
 & (4,632) & (4,521) & (4,489) \\
\midrule
State FE & No & Yes & No \\
State RE & No & No & Yes \\
$n$ & 765 & 765 & 765 \\
$R^2$ / Overall $R^2$ & 0.312 & 0.298 & 0.308 \\
\bottomrule
\end{tabular}
\begin{tablenotes}
\small
\item \textit{Notes}: Intercept-only benchmarking specification with year effects and no additional time-varying covariates. The FE column includes state fixed effects (not reported), and the RE column includes a state-level random intercept. Standard errors in parentheses (clustered at the state level for FE and RE). Coefficients are interpreted descriptively as a two-way effects decomposition of state/year variation rather than causal estimates of covariate effects. * $p < 0.05$; ** $p < 0.01$; *** $p < 0.001$.
\end{tablenotes}
\end{table}

\subsection{Cross-Sectional Allocation Model Specifications}
\label{app:gravity_models}

Table~\ref{tab:app_gravity} presents the complete cross-sectional allocation model estimates across the PPML specifications discussed in Section~\ref{subsec:gravity_results}. As noted in Section~\ref{subsec:gravity_methods}, these models omit origin--destination distance because variation in distance from a given origin country to different U.S.\ states is minimal.

\begin{table}[htbp]
\centering
\caption{Cross-Sectional Allocation Model Results: PPML Specifications}
\label{tab:app_gravity}
\begin{tabular}{@{}lccc@{}}
\toprule
\textbf{Variable} & \textbf{(1) Simple} & \textbf{(2) Full} & \textbf{(3) State FE} \\
\midrule
Log(Diaspora Stock) & 0.448*** & 0.140 & 0.163 \\
 & (0.104) & (0.248) & (0.148) \\
Log(Origin Stock in U.S.) & --- & 0.046 & --- \\
 & & (0.192) & \\
Log(State Foreign-Born Total) & --- & 0.792*** & --- \\
 & & (0.212) & \\
Constant & 1.267 & $-$7.711* & --- \\
 & (0.980) & (3.123) & \\
\midrule
Origin FE & No & No & No \\
Destination FE & No & No & Yes \\
$n$ & 4,845 & 4,845 & 4,845 \\
Pseudo $R^2$ & 0.236 & 0.399 & 0.413 \\
Log Likelihood & $-$934,340 & $-$734,629 & $-$717,292 \\
\bottomrule
\end{tabular}
\begin{tablenotes}
\small
\item \textit{Notes}: Poisson pseudo-maximum likelihood (PPML) estimation. Two-way clustered SEs (state and origin) in parentheses. Log covariates are computed as $\ln(x+1)$ (log1p). Dependent variable: LPR admissions by state-country pair, FY 2023. * $p < 0.05$; *** $p < 0.001$.
\end{tablenotes}
\end{table}

\subsection{Difference-in-Differences Specifications}
\label{app:did_models}

Table~\ref{tab:app_did} reports the complete difference-in-differences estimates for the Travel Ban analysis, including alternative specifications and robustness checks.

\begin{table}[htbp]
\centering
\caption{Difference-in-Differences Estimates: Travel Ban Policy-Associated Divergence}
\label{tab:app_did}
\begin{tabular}{@{}lcccc@{}}
\toprule
\textbf{Variable} & \textbf{(1) Basic} & \textbf{(2) + Year FE} & \textbf{(3) + Country FE} & \textbf{(4) Full} \\
\midrule
Affected $\times$ Post & $-$1.689** & $-$1.680** & $-$1.368* & $-$1.393* \\
 & (0.612) & (0.617) & (0.640) & (0.647) \\
Affected & 2.755** & 2.753** & --- & --- \\
 & (0.940) & (0.950) & & \\
Post & $-$0.221 & --- & $-$0.321 & --- \\
 & (0.206) & & (0.166) & \\
Constant & 3.392*** & 2.951*** & --- & --- \\
 & (0.247) & (0.338) & & \\
\midrule
Year FE & No & Yes & No & Yes \\
Country FE & No & No & Yes & Yes \\
$n$ & 1,119 & 1,119 & 1,119 & 1,119 \\
$R^2$ & 0.095 & 0.099 & 0.801 & 0.807 \\
Pre-trend test ($p$) & --- & --- & --- & 0.349 \\
\bottomrule
\end{tabular}
\begin{tablenotes}
\small
\item \textit{Notes}: Dependent variable: $\ln(\text{refugee arrivals} + 1)$. Standard errors clustered by nationality in parentheses. Affected countries: Iran, Iraq, Libya, Somalia, Sudan, Syria, Yemen. Post $= 1$ for years $\geq 2018$. * $p < 0.05$; ** $p < 0.01$; *** $p < 0.001$.
\end{tablenotes}
\end{table}

% ==============================================================================
% A.2 ROBUSTNESS CHECKS
% ==============================================================================

\section{Robustness Checks}
\label{app:robustness}

This section summarizes robustness checks for the main empirical findings.

\subsection{Alternative Unit Root Tests}
\label{app:unit_root_robustness}

Table~\ref{tab:app_unitroot} reports unit root test results using alternative specifications and a break-robust test, providing sensitivity checks for North Dakota international migration.

\begin{table}[htbp]
\centering
\caption{Robustness of Unit Root Findings}
\label{tab:app_unitroot}
\begin{tabular}{@{}llccc@{}}
\toprule
\textbf{Test} & \textbf{Specification} & \textbf{Statistic} & \textbf{$p$-value} & \textbf{Conclusion} \\
\midrule
ADF & Constant only & $-$1.453 & 0.556 & Fail to reject unit root \\
PP & Constant only & $-$0.620 & 0.867 & Fail to reject unit root \\
KPSS & Constant only & 0.323 & $\ge 0.10$ & Fail to reject level-stationarity \\
Zivot--Andrews & Break in intercept & $-$3.049 & 0.897 & Fail to reject unit root (break 2021) \\
\midrule
\multicolumn{5}{l}{\textit{First-differenced series}} \\
ADF & Constant only & $-$3.843 & 0.002 & Reject unit root \\
PP & Constant only & $-$4.012 & 0.001 & Reject unit root \\
KPSS & Constant only & 0.189 & $\ge 0.10$ & Fail to reject stationarity \\
\bottomrule
\end{tabular}
\begin{tablenotes}
\small
\item \textit{Notes}: ADF = Augmented Dickey-Fuller; PP = Phillips-Perron; KPSS = Kwiatkowski-Phillips-Schmidt-Shin. KPSS null is stationarity; statsmodels reports 0.10 when $p \ge 0.10$. Zivot--Andrews allows one endogenous break (intercept). Sample: 2010--2024 ($n = 15$).
\end{tablenotes}
\end{table}

\subsection{Vintage Diagnostic for the Long-Run PEP Series}
\label{app:pep_vintage_regime}

Figure~\ref{fig:app_pep_regime_diagnostic} visualizes the long-run (2000--2024) PEP net international migration series for North Dakota, explicitly separating decennial-vintage measurement periods and treating the COVID year (2020) as a one-year intervention. The purpose of this diagnostic is to clarify that the extended series is a \emph{spliced} measurement product, not a methodologically homogeneous time series, and to quantify vintage-specific trend slopes used in robustness discussions.

\begin{figure}[htbp]
\centering
\includegraphics[width=0.90\textwidth]{figures/fig_app_pep_regime_diagnostic.png}
\caption{Measurement-vintage diagnostic for North Dakota PEP net international migration, 2000--2024. Points are colored by PEP vintage (2009/2020/2024). The fitted line reflects a piecewise (vintage-period) trend model with a COVID-19 pulse intervention in 2020 and HAC (Newey--West) standard errors.}
\label{fig:app_pep_regime_diagnostic}
\end{figure}

\begin{table}[htbp]
\centering
\caption{Vintage-Specific Trend Slopes in Long-Run PEP Series (Diagnostic)}
\label{tab:app_pep_regime_slopes}
\small
\begin{tabular}{@{}lccc@{}}
\toprule
\textbf{Vintage (Years)} & \textbf{Slope} & \textbf{SE (HAC)} & \textbf{$p$-value} \\
\midrule
Vintage 2009 (2000--2009) & 39.2 & 17.8 & 0.028 \\
Vintage 2020 (2010--2019) & 72.4 & 72.3 & 0.316 \\
Vintage 2024 (2020--2024) & 1500.1 & 191.6 & $< 0.001$ \\
\bottomrule
\end{tabular}
\begin{tablenotes}
\small
\item \textit{Notes}: Slopes represent annual change in PEP net international migration (persons) within each decennial-vintage period. Standard errors are HAC (Newey--West) with 2 lags. The model includes level shifts at vintage boundaries and a one-year COVID-19 pulse intervention for 2020. Replication artifacts: \texttt{\detokenize{sdc_2024_replication/scripts/statistical_analysis/results/module_B1_pep_regime_modeling.json}} and \texttt{\detokenize{sdc_2024_replication/scripts/statistical_analysis/results/module_B1_pep_regime_modeling_slopes.csv}}.
\end{tablenotes}
\end{table}

\subsection{Covariate-Conditioned Near-Term Forecast Anchor (2025--2029)}
\label{app:covariate_anchor}

As an appendix-only robustness check, we estimate a parsimonious state-space ``local level + regression'' model that conditions North Dakota \PEP net international migration on lagged auxiliary signals: (i) ND refugee arrivals (RPC; fiscal-year totals) and (ii) ND LPR admissions (DHS; fiscal-year totals). We also report a sensitivity specification that adds an ACS moved-from-abroad proxy (B07007; ``Foreign born, moved from abroad''). This diagnostic is not used to parameterize the main scenario engine or to replace the Moderate baseline; it is intended to assess whether policy-linked covariates materially change the near-term forecast under minimal and transparent covariate-path assumptions (last-observation carried forward). Scope and interpretation are formalized in ADR-031.

\begin{figure}[htbp]
\centering
\includegraphics[width=0.90\textwidth]{figures/fig_app_covariate_anchor_2025_2029.png}
\caption{Appendix diagnostic: covariate-conditioned near-term forecast anchor (2025--2029). The baseline is the univariate ARIMA forecast used for the Moderate scenario's near-term anchor. The covariate-conditioned forecasts use a local-level state-space regression with lagged regressors (RPC refugee FY totals, DHS LPR FY totals; ACS B07007 sensitivity). Covariate paths beyond the last observed year are held constant (LOCF) to avoid embedding a separate long-horizon covariate-forecasting model. Shaded bands indicate 95\% prediction intervals for each forecast (baseline in gray; covariate-conditioned specifications in their corresponding colors).}
\label{fig:app_covariate_anchor_2025_2029}
\end{figure}

\begin{table}[htbp]
\centering
\caption{Near-Term Forecast Comparison: Baseline vs.\ Covariate-Conditioned Anchor (2025--2029)}
\label{tab:app_covariate_anchor_2025_2029}
\small
\begin{tabular}{@{}lccc@{}}
\toprule
\textbf{Year} & \textbf{Baseline (ARIMA)} & \textbf{Refugees + LPR (P0)} & \textbf{+ ACS Sensitivity (S3)} \\
\midrule
2025 & 5,126 [2,928, 7,324] & 5,968 [3,946, 7,989] & 4,542 [2,906, 6,179] \\
2026 & 5,126 [2,018, 8,234] & 5,968 [3,109, 8,826] & 4,542 [2,647, 6,437] \\
2027 & 5,126 [1,320, 8,932] & 5,968 [2,467, 9,468] & 4,542 [2,420, 6,665] \\
2028 & 5,126 [731, 9,521] & 5,968 [1,925, 10,010] & 4,542 [2,215, 6,870] \\
2029 & 5,126 [212, 10,040] & 5,968 [1,448, 10,487] & 4,542 [2,026, 7,059] \\
\bottomrule
\end{tabular}
\begin{tablenotes}
\small
\item \textit{Notes}: Entries report point forecasts with 95\% prediction intervals in brackets. The P0 and S3 columns are produced by the appendix state-space regression (local level + lagged covariates) and are presented as a diagnostic overlay; they do not replace the Moderate baseline or feed the main Monte Carlo scenario engine.
\end{tablenotes}
\end{table}

\subsection{Alternative Structural Break Tests}
\label{app:break_robustness}

The Bai-Perron procedure for endogenous break detection yields the following results under alternative information criteria:

\begin{itemize}
    \item \textbf{BIC criterion}: 0 breaks detected (penalty on parameters dominates)
    \item \textbf{AIC criterion}: 1 break detected at 2020 ($F = 14.23$, $p < 0.001$)
    \item \textbf{Sequential procedure}: 2 breaks detected at 2017 and 2020 (marginal significance at 2017)
\end{itemize}

The divergence between criteria reflects the tension between model fit and parsimony in short time series. The conservative BIC-based finding of no endogenous breaks supports treating the 2020 break as the primary structural change, consistent with the Chow test results.

	\subsection{VAR Model Results}
	\label{app:var_results}

	Section~\ref{subsec:time_series_methods} describes the VAR methodology, and Section~\ref{subsec:scenario_results} explains why multivariate models are treated as diagnostics rather than inputs to the scenario engine. This section reports the underlying VAR diagnostics and coefficient estimates.

\paragraph{Lag Selection and Model Fit.} Lag order selection across multiple criteria uniformly favors VAR(1):

\begin{center}
\small
\begin{tabular}{@{}lcc@{}}
\toprule
\textbf{Criterion} & \textbf{Optimal Lag} & \textbf{Value at Lag 1} \\
\midrule
AIC & 1 & 39.00 \\
BIC & 1 & 39.26 \\
HQIC & 1 & 38.95 \\
FPE & 1 & $8.83 \times 10^{16}$ \\
\bottomrule
\end{tabular}
\end{center}

The VAR(1) model achieves $R^2 = 0.58$ for the ND equation and $R^2 = 0.60$ for the US equation, with log-likelihood $-305.9$ and AIC = 38.87.

\paragraph{VAR Coefficient Estimates.} Table~\ref{tab:app_var} reports the VAR(1) coefficient matrix. No individual coefficient achieves statistical significance at the 5\% level, reflecting the very short sample ($n = 14$ after differencing for lag construction) and wide confidence intervals characteristic of small-sample VAR estimation.

\begin{table}[htbp]
\centering
\caption{VAR(1) Coefficient Estimates}
\label{tab:app_var}
\small
\begin{tabular}{@{}llcccc@{}}
\toprule
\textbf{Equation} & \textbf{Regressor} & \textbf{Coef.} & \textbf{SE} & \textbf{$t$} & \textbf{$p$} \\
\midrule
ND Int'l Mig. & L1.ND Int'l Mig. & 123.9 & 535.1 & 0.23 & 0.817 \\
 & L1.US Int'l Mig. & $-0.264$ & 0.660 & $-0.40$ & 0.689 \\
\midrule
US Int'l Mig. & L1.ND Int'l Mig. & 198,274 & 256,279 & 0.77 & 0.439 \\
 & L1.US Int'l Mig. & $-12.1$ & 315.9 & $-0.04$ & 0.969 \\
\bottomrule
\end{tabular}
\begin{tablenotes}
\small
\item \textit{Notes}: VAR(1) estimated on $n = 14$ annual observations (2011--2024). L1 = one-year lag. Coefficients represent units of dependent variable per unit of lagged regressor. No coefficient is statistically significant at $\alpha = 0.05$.
\end{tablenotes}
\end{table}

\paragraph{Granger Causality.} Granger causality tests yield mixed conclusions depending on the test statistic employed:

\begin{itemize}
    \item US $\rightarrow$ ND: The $\chi^2$ test marginally rejects the null ($\chi^2 = 4.10$, $p = 0.043$), but the $F$-test does not ($F = 3.22$, $p = 0.100$). This discrepancy reflects the small-sample size where asymptotic $\chi^2$ approximations may be unreliable.
    \item ND $\rightarrow$ US: All tests fail to reject the null ($p > 0.96$), as expected given ND's negligible share of national migration.
\end{itemize}

\paragraph{Forecast Error Variance Decomposition.} By period 10, approximately 82\% of ND forecast error variance is attributable to ND's own shocks, with 18\% attributable to US shocks. This proportion stabilizes after period 4, suggesting that short-horizon forecasts are dominated by ND-specific innovations while longer horizons incorporate modest US spillovers.

\paragraph{Cointegration.} Engle-Granger and Johansen tests yield conflicting results. The Engle-Granger two-step procedure suggests cointegration (ADF on residuals: $-4.48$, $p < 0.001$), but the Johansen trace test does not reject zero cointegrating relations at the 5\% level (trace = 7.18, critical value = 15.49). Given the mixed integration orders of the component series (ND: I(1); US: I(0) per ADF) and the very short sample, we do not impose a cointegrating restriction and proceed with VAR in levels.

	\paragraph{Role in Forecasting.} The VAR is not used for standalone long-horizon forecasting because it requires future values of US international migration---information unavailable at forecast time. The VAR's contribution to the analysis is primarily diagnostic: it characterizes ND--US comovement and provides variance decomposition context for interpreting scenario uncertainty.

\subsection{Sensitivity to PPML Specification}
\label{app:ppml_robustness}

Table~\ref{tab:app_ppml} reports the diaspora coefficient from the full cross-sectional allocation model under alternative covariance estimators, highlighting the inflation of uncertainty under clustered inference.

\begin{table}[htbp]
\centering
\caption{Diaspora Association Sensitivity to Covariance Estimator}
\label{tab:app_ppml}
\begin{tabular}{@{}lcc@{}}
\toprule
\textbf{Covariance Estimator} & \textbf{Coefficient} & \textbf{SE} \\
\midrule
Model-based (Poisson MLE) & 0.140 & 0.002 \\
HC1 robust & 0.140 & 0.176 \\
Clustered by state & 0.140 & 0.113 \\
Clustered by origin & 0.140 & 0.282 \\
Two-way clustered (state $\times$ origin) & 0.140 & 0.248 \\
\bottomrule
\end{tabular}
\begin{tablenotes}
\small
\item \textit{Notes}: Estimates come from the full gravity specification in Table~\ref{tab:app_gravity}. Clustered SEs account for correlation within destinations and origins.
\end{tablenotes}
\end{table}

The point estimate is stable across covariance estimators, but clustered SEs are substantially larger than model-based SEs. We therefore report two-way clustered inference in the main text.

\subsection{DiD Parallel Trends Robustness}
\label{app:parallel_trends}

The parallel trends assumption is evaluated through multiple approaches:

\begin{enumerate}
    \item \textbf{Pre-treatment trend test}: The interaction between Affected and a linear pre-treatment trend yields 0.087 ($t = 0.97$, $p = 0.334$). The test fails to reject parallel trends.

    \item \textbf{Placebo treatment dates}: Placebo tests shifting the treatment date to 2015 or 2016 were conducted in the prior HC3 specification; these checks should be re-run under clustered inference.

    \item \textbf{Event study plot}: Figure~\ref{fig:eventstudy} shows negative post-treatment coefficients in 2018--2019, but the joint pre-treatment test rejects parallel trends when the full pre-period is included ($p < 0.001$), indicating longer-run divergence.
\end{enumerate}

\subsection{Scenario Arithmetic}
\label{app:scenario_arithmetic}

This section documents the update rules used to generate scenario paths in Module 9.

\paragraph{CBO (Jan 2026).} This scenario scales the Congressional Budget Office's January 2026 national net immigration projection to North Dakota using the state's mean share of U.S.\ net international migration in the \PEP series (2010--2024):
\begin{verbatim}
nd_cbo_t = share_nd * cbo_us_t   (t = 2025..2045)
\end{verbatim}

\paragraph{Pre-2020 Trend.} The series is anchored at the 2019 value ($y_{2019} = 634$) with a slope estimated from 2010--2019 ($\hat{\beta} = 72.43$ per year):
\begin{verbatim}
pre2020_t = y_2019 + beta * (t - 2019)
\end{verbatim}

\paragraph{Moderate.} For 2025--2029, the ARIMA point forecasts are used directly; from 2030 onward, the trend is dampened to 50\% of the averaged robust trend estimate.

\paragraph{Zero.} Net international migration is set to zero for all years.

\subsection{Monte Carlo Uncertainty and Wave Simulation}
\label{app:monte_carlo_variance}

The Monte Carlo simulation in Module 9 combines two sources of stochastic variation: (1) ARIMA baseline uncertainty from the estimated random-walk process, and (2) wave duration draws from the Cox proportional hazards model (Module 8). A potential concern is double-counting of refugee-driven volatility, since the ARIMA model is trained on the full \PEP net migration series---which includes historical refugee arrivals---while wave simulations add stochastic refugee wave contributions on top of this baseline.

We address this concern as follows. The wave simulation does not add independent refugee variance; rather, it \emph{modulates} the timing and persistence of above-baseline arrivals conditional on an active wave being detected. Specifically:
\begin{enumerate}
    \item The ARIMA baseline captures the unconditional mean and variance of the \PEP series, which reflects average historical refugee influence.
    \item The Cox-based wave simulation draws wave durations conditional on observed wave characteristics (intensity, age, origin region) and applies a lifecycle shape function (initiation $\rightarrow$ peak $\rightarrow$ decline) to allocate excess arrivals over time.
    \item Wave contributions are defined as \emph{excess} arrivals above the origin-specific baseline, not as additive draws independent of the ARIMA process.
\end{enumerate}

Nevertheless, because the ARIMA was not trained on a refugee-stripped series, some overlap between baseline variance and wave-induced variance may remain. This design choice was made for parsimony and because separating refugee from non-refugee components in \PEP data is infeasible without auxiliary assumptions. Consequently, the Monte Carlo prediction intervals should be interpreted as \emph{conservative} (potentially inflated) rather than calibrated probability statements. This limitation is acknowledged in Section~\ref{subsec:scenario_results}, where we note that the 95\% prediction interval spans a factor of 4.1 by 2045.

% ==============================================================================
% A.3 DATA SOURCES
% ==============================================================================

\section{Data Sources}
\label{app:data_sources}

This section provides detailed documentation of all data sources employed in the analysis.

\subsection{Census Bureau Population Estimates Program}
\label{app:pep}

\begin{itemize}
    \item \textbf{Source}: U.S. Census Bureau, Population Estimates Program
    \item \textbf{Vintage}: 2024
    \item \textbf{Coverage}: 2010--2024 annual estimates
    \item \textbf{Geographic scope}: 50 states plus District of Columbia
    \item \textbf{Variables used}: INTERNATIONALMIG (net international migration component)
    \item \textbf{URL}: \url{https://www.census.gov/programs-surveys/popest.html}
    \item \textbf{Access date}: December 2024
\end{itemize}

\subsection{Department of Homeland Security LPR Statistics}
\label{app:dhs}

\begin{itemize}
    \item \textbf{Source}: Department of Homeland Security, Office of Immigration Statistics
    \item \textbf{Publication}: Yearbook of Immigration Statistics
    \item \textbf{Coverage}: Fiscal Year 2023
    \item \textbf{Geographic scope}: State of intended residence $\times$ country of birth
    \item \textbf{Variables used}: LPR admissions count
    \item \textbf{URL}: \url{https://www.dhs.gov/immigration-statistics/yearbook}
    \item \textbf{Access date}: December 2024
\end{itemize}

\subsection{American Community Survey}
\label{app:acs}

\begin{itemize}
    \item \textbf{Source}: U.S. Census Bureau, American Community Survey
    \item \textbf{Vintage}: 5-year estimates, 2009--2023
    \item \textbf{Geographic scope}: State level
    \item \textbf{Tables used}: B05006 (Place of Birth for the Foreign-Born Population)
    \item \textbf{Variables used}: Foreign-born population by country/region of birth
    \item \textbf{URL}: \url{https://data.census.gov/}
    \item \textbf{Access date}: December 2024
\end{itemize}

\subsection{Refugee Processing Center}
\label{app:rpc}

\begin{itemize}
    \item \textbf{Source}: Department of State, Refugee Processing Center
    \item \textbf{Coverage}: Fiscal Years 2002--2020
    \item \textbf{Geographic scope}: State of initial resettlement $\times$ nationality
    \item \textbf{Variables used}: Refugee arrivals count
    \item \textbf{URL}: \url{https://www.wrapsnet.org/}
    \item \textbf{Access date}: December 2024
\end{itemize}

\subsection{Ancillary Data}
\label{app:ancillary}

\begin{itemize}
    \item \textbf{World Bank}: Origin country population (World Development Indicators)
    \item \textbf{CEPII}: Geographic distances between countries (GeoDist database)
    \item \textbf{UN Population Division}: World population estimates and projections
\end{itemize}

% ==============================================================================
% A.4 SUPPLEMENTARY FIGURES
% ==============================================================================

\section{Supplementary Figures}
\label{app:figures}

\subsection{State-Level Migration Distribution}

Figure~\ref{fig:app_state_distribution} displays the distribution of \PEP net international migration across U.S. states, highlighting North Dakota's position in the lower tail.

\begin{figure}[htbp]
\centering
\includegraphics[width=0.9\textwidth]{figures/fig_app_state_distribution.pdf}
\caption{Distribution of mean annual net international migration (\PEP) across U.S.\ states, 2010--2024. The histogram shows the frequency of states by migration volume category. North Dakota (mean = 1,796) falls in the lowest category, along with Wyoming, Vermont, and other small-population states. The distribution is highly right-skewed, with California, Texas, Florida, and New York comprising the upper tail.}
\label{fig:app_state_distribution}
\end{figure}

\subsection{Extended Travel Ban Event Study (Supplemental)}

Figure~\ref{fig:app_event_study_extended} extends the nationality-level Travel Ban event study through FY2024. Coefficients after FY2019 should be interpreted as treated--control divergence across overlapping shocks and policy periods (COVID-19 disruptions, formal rescission in 2021, and USRAP rebuilding), not as a causal Travel Ban effect.

\begin{figure}[htbp]
\centering
\includegraphics[width=0.9\textwidth]{figures/fig_app_event_study_extended.pdf}
\caption{Extended event study coefficients for treated Travel Ban nationalities (Iran, Iraq, Libya, Somalia, Sudan, Syria, Yemen) relative to control nationalities, using FY2017 as the reference period ($t=-1$). Shaded bands mark post-2017 policy-period blocks for visual context only. This figure is a supplemental descriptive extension: post-2020 estimates do not represent a standalone Travel Ban causal effect because FY2020--FY2024 overlap multiple confounding shocks and the Travel Ban was formally rescinded in 2021.}
\label{fig:app_event_study_extended}
\end{figure}

\subsection{Residual Diagnostics}

Figure~\ref{fig:app_residuals} presents residual diagnostic plots for the ARIMA(0,1,0) specification.

\begin{figure}[htbp]
\centering
\includegraphics[width=0.9\textwidth]{figures/fig_app_residuals.pdf}
\caption{Residual diagnostics for ARIMA(0,1,0) model of North Dakota net international migration (\PEP). Panel~(A): Residual time series with no apparent pattern. Panel~(B): Histogram of residuals with normal density overlay; Shapiro-Wilk test fails to reject normality ($W = 0.968$, $p = 0.820$). Panel~(C): Q-Q plot showing approximate adherence to normal quantiles. Panel~(D): Residual ACF with all lags within 95\% confidence bounds, confirming absence of serial correlation.}
\label{fig:app_residuals}
\end{figure}

\subsection{Cox Model Diagnostics}

Figure~\ref{fig:app_schoenfeld} displays Schoenfeld residual plots for evaluating the proportional hazards assumption in the Cox regression model.

\begin{figure}[htbp]
\centering
\includegraphics[width=0.9\textwidth]{figures/fig_app_schoenfeld.pdf}
\caption{Schoenfeld residuals versus time for the Cox proportional hazards model of immigration wave duration. Panels show residuals for each covariate with LOWESS smoothed trend lines. Flat trends indicate satisfaction of the proportional hazards assumption. The global test fails to reject proportional hazards ($\chi^2 = 8.34$, $p = 0.214$).}
\label{fig:app_schoenfeld}
\end{figure}

% ==============================================================================
% A.5 VARIABLE DEFINITIONS
% ==============================================================================

\section{Variable Definitions}
\label{app:variables}

Table~\ref{tab:app_variables} provides formal definitions for all variables employed in the analysis.

\begin{table}[htbp]
\centering
\caption{Variable Definitions}
\label{tab:app_variables}
\begin{tabular}{@{}>{\raggedright\arraybackslash}p{3.5cm} >{\raggedright\arraybackslash}p{10cm}@{}}
\toprule
\textbf{Variable} & \textbf{Definition} \\
\midrule
Net International Migration (\PEP) & Net annual international migration component of change for state of residence (Census \PEP definition) \\
LPR Admissions (DHS, FY) & Count of lawful permanent resident admissions by state of intended residence and country of birth (fiscal year) \\
Diaspora Stock (\ACS) & Foreign-born population from origin country $o$ residing in destination state $d$ (\ACS estimate) \\
Location Quotient & Ratio of origin $o$'s share in state $d$ to origin $o$'s share nationally \\
HHI & Herfindahl-Hirschman Index: $\sum_i s_i^2 \times 10,000$ where $s_i \in [0,1]$ is origin $i$'s share \\
Wave & Period of $\geq 2$ consecutive years with arrivals $> 150\%$ of baseline \\
Wave Intensity & Peak-to-baseline ratio during wave \\
Affected Country & Travel Ban target: Iran, Iraq, Libya, Somalia, Sudan, Syria, Yemen \\
Post & Indicator for years $\geq 2018$ (first full year after Travel Ban implementation) \\
Shift-share (Bartik) index & $\sum_o \omega_{od,t_0} \cdot g_{o,t}^{\text{US},-d}$: $\omega_{od,t_0}$ is state $d$'s share of national arrivals from origin $o$ in the baseline year $t_0$ (here $t_0 = 2010$) and $g_{o,t}^{\text{US},-d}$ is the corresponding leave-one-out level change between $t_0$ and $t$ \\
\bottomrule
\end{tabular}
\end{table}

% ==============================================================================
% END APPENDIX
% ==============================================================================
