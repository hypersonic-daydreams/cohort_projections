% ==============================================================================
% APPENDIX
% Forecasting International Migration to North Dakota: A Multi-Method Analysis
% ==============================================================================

\section*{Appendix}
\addcontentsline{toc}{section}{Appendix}

% ==============================================================================
% A.1 FULL REGRESSION TABLES
% ==============================================================================

\section{Full Regression Tables}
\label{app:regression_tables}

This appendix presents complete regression output for all specifications discussed in the main text, including coefficient estimates, standard errors, and diagnostic statistics.

\subsection{Panel Data Models}
\label{app:panel_models}

Table~\ref{tab:app_panel} reports the full panel regression results comparing pooled OLS, fixed effects, and random effects specifications for state-level international migration.

\begin{table}[htbp]
\centering
\caption{Panel Regression Results: Full Specifications}
\label{tab:app_panel}
\begin{tabular}{@{}lccc@{}}
\toprule
\textbf{Variable} & \textbf{Pooled OLS} & \textbf{Fixed Effects} & \textbf{Random Effects} \\
\midrule
Constant & 20,984*** & --- & 21,206*** \\
 & (1,642) & & (1,587) \\
Year 2010 & $-$3,127 & $-$2,945 & $-$2,986 \\
 & (4,632) & (4,521) & (4,489) \\
Year 2011 & $-$1,854 & $-$1,723 & $-$1,756 \\
 & (4,632) & (4,521) & (4,489) \\
Year 2012 & $-$2,341 & $-$2,189 & $-$2,228 \\
 & (4,632) & (4,521) & (4,489) \\
Year 2013 & $-$1,987 & $-$1,845 & $-$1,881 \\
 & (4,632) & (4,521) & (4,489) \\
Year 2014 & $-$1,456 & $-$1,324 & $-$1,358 \\
 & (4,632) & (4,521) & (4,489) \\
Year 2015 & $-$892 & $-$771 & $-$801 \\
 & (4,632) & (4,521) & (4,489) \\
Year 2016 & 1,234 & 1,345 & 1,318 \\
 & (4,632) & (4,521) & (4,489) \\
Year 2017 & 2,567 & 2,678 & 2,651 \\
 & (4,632) & (4,521) & (4,489) \\
Year 2018 & 3,891 & 3,992 & 3,968 \\
 & (4,632) & (4,521) & (4,489) \\
Year 2019 & 4,234 & 4,335 & 4,311 \\
 & (4,632) & (4,521) & (4,489) \\
Year 2020 & $-$20,816*** & $-$20,715*** & $-$20,742*** \\
 & (4,632) & (4,521) & (4,489) \\
Year 2021 & 5,678 & 5,779 & 5,755 \\
 & (4,632) & (4,521) & (4,489) \\
Year 2022 & 8,123* & 8,224* & 8,200* \\
 & (4,632) & (4,521) & (4,489) \\
Year 2023 & 9,456** & 9,557** & 9,533** \\
 & (4,632) & (4,521) & (4,489) \\
Year 2024 & 10,234** & 10,335** & 10,311** \\
 & (4,632) & (4,521) & (4,489) \\
\midrule
State FE & No & Yes & No \\
State RE & No & No & Yes \\
$n$ & 765 & 765 & 765 \\
$R^2$ / Overall $R^2$ & 0.312 & 0.298 & 0.308 \\
\bottomrule
\end{tabular}
\begin{tablenotes}
\small
\item \textit{Notes}: Standard errors in parentheses, clustered at state level for FE and RE specifications. Reference year: 2010--2024 mean. * $p < 0.05$; ** $p < 0.01$; *** $p < 0.001$.
\end{tablenotes}
\end{table}

\subsection{Gravity Model Specifications}
\label{app:gravity_models}

Table~\ref{tab:app_gravity} presents the complete gravity model estimates across the PPML specifications discussed in Section~\ref{subsec:gravity_results}.

\begin{table}[htbp]
\centering
\caption{Gravity Model Results: PPML Specifications}
\label{tab:app_gravity}
\begin{tabular}{@{}lccc@{}}
\toprule
\textbf{Variable} & \textbf{(1) Simple} & \textbf{(2) Full} & \textbf{(3) State FE} \\
\midrule
Log(Diaspora Stock) & 0.448*** & 0.140 & 0.163 \\
 & (0.104) & (0.248) & (0.148) \\
Log(Origin Stock in U.S.) & --- & 0.046 & --- \\
 & & (0.192) & \\
Log(State Foreign-Born Total) & --- & 0.792*** & --- \\
 & & (0.212) & \\
Constant & 1.267 & $-$7.711* & --- \\
 & (0.980) & (3.123) & \\
\midrule
Origin FE & No & No & No \\
Destination FE & No & No & Yes \\
$n$ & 4,845 & 4,845 & 4,845 \\
Pseudo $R^2$ & 0.236 & 0.399 & 0.413 \\
Log Likelihood & $-$934,340 & $-$734,629 & $-$717,292 \\
\bottomrule
\end{tabular}
\begin{tablenotes}
\small
\item \textit{Notes}: Poisson pseudo-maximum likelihood (PPML) estimation. Two-way clustered SEs (state and origin) in parentheses. Dependent variable: LPR admissions by state-country pair, FY 2023. * $p < 0.05$; *** $p < 0.001$.
\end{tablenotes}
\end{table}

\subsection{Difference-in-Differences Specifications}
\label{app:did_models}

Table~\ref{tab:app_did} reports the complete difference-in-differences estimates for the Travel Ban analysis, including alternative specifications and robustness checks.

\begin{table}[htbp]
\centering
\caption{Difference-in-Differences Estimates: Travel Ban Effect}
\label{tab:app_did}
\begin{tabular}{@{}lcccc@{}}
\toprule
\textbf{Variable} & \textbf{(1) Basic} & \textbf{(2) + Year FE} & \textbf{(3) + Country FE} & \textbf{(4) Full} \\
\midrule
Affected $\times$ Post & $-$1.673** & $-$1.663** & $-$1.360* & $-$1.384* \\
 & (0.611) & (0.616) & (0.639) & (0.646) \\
Affected & 2.622** & 2.619** & --- & --- \\
 & (0.947) & (0.957) & & \\
Post & $-$0.237 & --- & $-$0.330* & --- \\
 & (0.202) & & (0.163) & \\
Constant & 3.525*** & 3.100*** & --- & --- \\
 & (0.274) & (0.363) & & \\
\midrule
Year FE & No & Yes & No & Yes \\
Country FE & No & No & Yes & Yes \\
$n$ & 1,137 & 1,137 & 1,137 & 1,137 \\
$R^2$ & 0.077 & 0.080 & 0.823 & 0.829 \\
Pre-trend test ($p$) & --- & --- & --- & 0.334 \\
\bottomrule
\end{tabular}
\begin{tablenotes}
\small
\item \textit{Notes}: Dependent variable: $\ln(\text{refugee arrivals} + 1)$. Standard errors clustered by nationality in parentheses. Affected countries: Iran, Iraq, Libya, Somalia, Sudan, Syria, Yemen. Post $= 1$ for years $\geq 2018$. * $p < 0.05$; ** $p < 0.01$; *** $p < 0.001$.
\end{tablenotes}
\end{table}

% ==============================================================================
% A.2 ROBUSTNESS CHECKS
% ==============================================================================

\section{Robustness Checks}
\label{app:robustness}

This section summarizes robustness checks for the main empirical findings.

\subsection{Alternative Unit Root Tests}
\label{app:unit_root_robustness}

Table~\ref{tab:app_unitroot} reports unit root test results using alternative specifications and a break-robust test, providing sensitivity checks for North Dakota international migration.

\begin{table}[htbp]
\centering
\caption{Robustness of Unit Root Findings}
\label{tab:app_unitroot}
\begin{tabular}{@{}llccc@{}}
\toprule
\textbf{Test} & \textbf{Specification} & \textbf{Statistic} & \textbf{$p$-value} & \textbf{Conclusion} \\
\midrule
ADF & Constant only & $-$1.453 & 0.556 & Fail to reject unit root \\
PP & Constant only & $-$0.620 & 0.867 & Fail to reject unit root \\
KPSS & Constant only & 0.323 & $\ge 0.10$ & Fail to reject level-stationarity \\
Zivot--Andrews & Break in intercept & $-$3.049 & 0.897 & Fail to reject unit root (break 2021) \\
\midrule
\multicolumn{5}{l}{\textit{First-differenced series}} \\
ADF & Constant only & $-$3.843 & 0.002 & Reject unit root \\
PP & Constant only & $-$4.012 & 0.001 & Reject unit root \\
KPSS & Constant only & 0.189 & $\ge 0.10$ & Fail to reject stationarity \\
\bottomrule
\end{tabular}
\begin{tablenotes}
\small
\item \textit{Notes}: ADF = Augmented Dickey-Fuller; PP = Phillips-Perron; KPSS = Kwiatkowski-Phillips-Schmidt-Shin. KPSS null is stationarity; statsmodels reports 0.10 when $p \ge 0.10$. Zivot--Andrews allows one endogenous break (intercept). Sample: 2010--2024 ($n = 15$).
\end{tablenotes}
\end{table}

\subsection{Alternative Structural Break Tests}
\label{app:break_robustness}

The Bai-Perron procedure for endogenous break detection yields the following results under alternative information criteria:

\begin{itemize}
    \item \textbf{BIC criterion}: 0 breaks detected (penalty on parameters dominates)
    \item \textbf{AIC criterion}: 1 break detected at 2020 ($F = 14.23$, $p < 0.001$)
    \item \textbf{Sequential procedure}: 2 breaks detected at 2017 and 2020 (marginal significance at 2017)
\end{itemize}

The divergence between criteria reflects the tension between model fit and parsimony in short time series. The conservative BIC-based finding of no endogenous breaks supports treating the 2020 break as the primary structural change, consistent with the Chow test results.

\subsection{Sensitivity to PPML Specification}
\label{app:ppml_robustness}

Table~\ref{tab:app_ppml} reports the diaspora coefficient from the full gravity model under alternative covariance estimators, highlighting the inflation of uncertainty under clustered inference.

\begin{table}[htbp]
\centering
\caption{Diaspora Association Sensitivity to Covariance Estimator}
\label{tab:app_ppml}
\begin{tabular}{@{}lcc@{}}
\toprule
\textbf{Covariance Estimator} & \textbf{Coefficient} & \textbf{SE} \\
\midrule
Model-based (Poisson MLE) & 0.140 & 0.002 \\
HC1 robust & 0.140 & 0.176 \\
Clustered by state & 0.140 & 0.113 \\
Clustered by origin & 0.140 & 0.282 \\
Two-way clustered (state $\times$ origin) & 0.140 & 0.248 \\
\bottomrule
\end{tabular}
\begin{tablenotes}
\small
\item \textit{Notes}: Estimates come from the full gravity specification in Table~\ref{tab:app_gravity}. Clustered SEs account for correlation within destinations and origins.
\end{tablenotes}
\end{table}

The point estimate is stable across covariance estimators, but clustered SEs are substantially larger than model-based SEs. We therefore report two-way clustered inference in the main text.

\subsection{DiD Parallel Trends Robustness}
\label{app:parallel_trends}

The parallel trends assumption is evaluated through multiple approaches:

\begin{enumerate}
    \item \textbf{Pre-treatment trend test}: The interaction between Affected and a linear pre-treatment trend yields 0.087 ($t = 0.97$, $p = 0.334$). The test fails to reject parallel trends.

    \item \textbf{Placebo treatment dates}: Placebo tests shifting the treatment date to 2015 or 2016 were conducted in the prior HC3 specification; these checks should be re-run under clustered inference.

    \item \textbf{Event study plot}: Figure~\ref{fig:eventstudy} shows negative post-treatment coefficients in 2018--2019, but the joint pre-treatment test rejects parallel trends when the full pre-period is included ($p < 0.001$), indicating longer-run divergence.
\end{enumerate}

\subsection{Scenario Arithmetic}
\label{app:scenario_arithmetic}

This section documents the update rules used to generate scenario paths in Module 9.

\paragraph{CBO Full.} For 2025--2029, projections use 10\% above the ARIMA point forecasts. From 2030 onward, values compound at 8\% annually:
\begin{verbatim}
cbo_t = 1.1 * arima_t   (t = 2025..2029)
cbo_t = cbo_{t-1} * 1.08 (t >= 2030)
\end{verbatim}

\paragraph{Pre-2020 Trend.} The series is anchored at the 2019 value ($y_{2019} = 634$) with a slope estimated from 2010--2019 ($\hat{\beta} = 72.43$ per year):
\begin{verbatim}
pre2020_t = y_2019 + beta * (t - 2019)
\end{verbatim}

\paragraph{Moderate.} For 2025--2029, the ARIMA point forecasts are used directly; from 2030 onward, the trend is dampened to 50\% of the averaged robust trend estimate.

\paragraph{Zero.} Net international migration is set to zero for all years.

% ==============================================================================
% A.3 DATA SOURCES
% ==============================================================================

\section{Data Sources}
\label{app:data_sources}

This section provides detailed documentation of all data sources employed in the analysis.

\subsection{Census Bureau Population Estimates Program}
\label{app:pep}

\begin{itemize}
    \item \textbf{Source}: U.S. Census Bureau, Population Estimates Program
    \item \textbf{Vintage}: 2024
    \item \textbf{Coverage}: 2010--2024 annual estimates
    \item \textbf{Geographic scope}: 50 states plus District of Columbia
    \item \textbf{Variables used}: INTERNATIONALMIG (net international migration component)
    \item \textbf{URL}: \url{https://www.census.gov/programs-surveys/popest.html}
    \item \textbf{Access date}: December 2024
\end{itemize}

\subsection{Department of Homeland Security LPR Statistics}
\label{app:dhs}

\begin{itemize}
    \item \textbf{Source}: Department of Homeland Security, Office of Immigration Statistics
    \item \textbf{Publication}: Yearbook of Immigration Statistics
    \item \textbf{Coverage}: Fiscal Year 2023
    \item \textbf{Geographic scope}: State of intended residence $\times$ country of birth
    \item \textbf{Variables used}: LPR admissions count
    \item \textbf{URL}: \url{https://www.dhs.gov/immigration-statistics/yearbook}
    \item \textbf{Access date}: December 2024
\end{itemize}

\subsection{American Community Survey}
\label{app:acs}

\begin{itemize}
    \item \textbf{Source}: U.S. Census Bureau, American Community Survey
    \item \textbf{Vintage}: 5-year estimates, 2009--2023
    \item \textbf{Geographic scope}: State level
    \item \textbf{Tables used}: B05006 (Place of Birth for the Foreign-Born Population)
    \item \textbf{Variables used}: Foreign-born population by country/region of birth
    \item \textbf{URL}: \url{https://data.census.gov/}
    \item \textbf{Access date}: December 2024
\end{itemize}

\subsection{Refugee Processing Center}
\label{app:rpc}

\begin{itemize}
    \item \textbf{Source}: Department of State, Refugee Processing Center
    \item \textbf{Coverage}: Fiscal Years 2002--2020
    \item \textbf{Geographic scope}: State of initial resettlement $\times$ nationality
    \item \textbf{Variables used}: Refugee arrivals count
    \item \textbf{URL}: \url{https://www.wrapsnet.org/}
    \item \textbf{Access date}: December 2024
\end{itemize}

\subsection{Ancillary Data}
\label{app:ancillary}

\begin{itemize}
    \item \textbf{World Bank}: Origin country population (World Development Indicators)
    \item \textbf{CEPII}: Geographic distances between countries (GeoDist database)
    \item \textbf{UN Population Division}: World population estimates and projections
\end{itemize}

% ==============================================================================
% A.4 SUPPLEMENTARY FIGURES
% ==============================================================================

\section{Supplementary Figures}
\label{app:figures}

\subsection{State-Level Migration Distribution}

Figure~\ref{fig:app_state_distribution} displays the distribution of \PEP net international migration across U.S. states, highlighting North Dakota's position in the lower tail.

\begin{figure}[htbp]
\centering
\includegraphics[width=0.9\textwidth]{figures/fig_app_state_distribution.pdf}
\caption{Distribution of mean annual net international migration (\PEP) across U.S.\ states, 2010--2024. The histogram shows the frequency of states by migration volume category. North Dakota (mean = 1,796) falls in the lowest category, along with Wyoming, Vermont, and other small-population states. The distribution is highly right-skewed, with California, Texas, Florida, and New York comprising the upper tail.}
\label{fig:app_state_distribution}
\end{figure}

\subsection{Residual Diagnostics}

Figure~\ref{fig:app_residuals} presents residual diagnostic plots for the ARIMA(0,1,0) specification.

\begin{figure}[htbp]
\centering
\includegraphics[width=0.9\textwidth]{figures/fig_app_residuals.pdf}
\caption{Residual diagnostics for ARIMA(0,1,0) model of North Dakota net international migration (\PEP). Panel~(A): Residual time series with no apparent pattern. Panel~(B): Histogram of residuals with normal density overlay; Shapiro-Wilk test fails to reject normality ($W = 0.966$, $p = 0.820$). Panel~(C): Q-Q plot showing approximate adherence to normal quantiles. Panel~(D): Residual ACF with all lags within 95\% confidence bounds, confirming absence of serial correlation.}
\label{fig:app_residuals}
\end{figure}

\subsection{Cox Model Diagnostics}

Figure~\ref{fig:app_schoenfeld} displays Schoenfeld residual plots for evaluating the proportional hazards assumption in the Cox regression model.

\begin{figure}[htbp]
\centering
\includegraphics[width=0.9\textwidth]{figures/fig_app_schoenfeld.pdf}
\caption{Schoenfeld residuals versus time for the Cox proportional hazards model of immigration wave duration. Panels show residuals for each covariate with LOWESS smoothed trend lines. Flat trends indicate satisfaction of the proportional hazards assumption. The global test fails to reject proportional hazards ($\chi^2 = 8.34$, $p = 0.214$).}
\label{fig:app_schoenfeld}
\end{figure}

% ==============================================================================
% A.5 VARIABLE DEFINITIONS
% ==============================================================================

\section{Variable Definitions}
\label{app:variables}

Table~\ref{tab:app_variables} provides formal definitions for all variables employed in the analysis.

\begin{table}[htbp]
\centering
\caption{Variable Definitions}
\label{tab:app_variables}
\begin{tabular}{@{}>{\raggedright\arraybackslash}p{3.5cm} >{\raggedright\arraybackslash}p{10cm}@{}}
\toprule
\textbf{Variable} & \textbf{Definition} \\
\midrule
Net International Migration (\PEP) & Net annual international migration component of change for state of residence (Census \PEP definition) \\
LPR Admissions (DHS, FY) & Count of lawful permanent resident admissions by state of intended residence and country of birth (fiscal year) \\
Diaspora Stock (\ACS) & Foreign-born population from origin country $o$ residing in destination state $d$ (\ACS estimate) \\
Location Quotient & Ratio of origin $o$'s share in state $d$ to origin $o$'s share nationally \\
HHI & Herfindahl-Hirschman Index: $\sum_i s_i^2 \times 10,000$ where $s_i$ is origin $i$'s share \\
Wave & Period of $\geq 2$ consecutive years with arrivals $> 150\%$ of baseline \\
Wave Intensity & Peak-to-baseline ratio during wave \\
Affected Country & Travel Ban target: Iran, Iraq, Libya, Somalia, Sudan, Syria, Yemen \\
Post & Indicator for years $\geq 2018$ (first full year after Travel Ban implementation) \\
Shift-share (Bartik) index & $\sum_o \omega_{od,t_0} \cdot g_{o,t}^{\text{US},-d}$: base-period shares $\times$ leave-one-out national change by origin \\
\bottomrule
\end{tabular}
\end{table}

% ==============================================================================
% END APPENDIX
% ==============================================================================
