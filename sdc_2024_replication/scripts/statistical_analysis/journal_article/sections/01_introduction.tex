% ==============================================================================
% INTRODUCTION
% Forecasting International Migration to North Dakota: A Multi-Method Analysis
% ==============================================================================

\section{Introduction}
\label{sec:introduction}

% ------------------------------------------------------------------------------
% Opening Hook: ND's Unique Migration Position
% ------------------------------------------------------------------------------

Most accounts of U.S.\ immigration are written from the perspective of scale: national totals, gateway metropolitan areas, and datasets large enough to make uncertainty feel like an afterthought. That emphasis made sense when national population growth was robust and local idiosyncrasies averaged out. Today, as fertility declines and populations age, the subnational periphery matters. In many small-population states, whether the population grows, stagnates, or declines increasingly hinges on a migration component that is tiny in national arithmetic but large in local consequence.

North Dakota occupies a distinctive position in this landscape. With roughly 797,000 residents (47th among U.S.\ states; July 1, 2024 \PEP estimate \citep{CensusPEP2024}), the state receives about 0.17\% of national international migration flows---below its 0.23\% share of the U.S.\ population \citep{SmallAreaForecasting2021}. Yet this modest arithmetic masks a demographic reality of considerable consequence. In a state where natural increase is weakening and domestic out-migration has been persistent, international migration functions as a demographic swing factor: a few hundred additional arrivals can move school enrollment, housing demand, labor supply, and service needs in ways that would be statistically invisible in larger states.

The Great Plains has experienced sustained population decline throughout much of the twentieth and early twenty-first centuries, a demographic trajectory shaped by agricultural consolidation, limited economic diversification, and the magnetic pull of metropolitan opportunity elsewhere \citep{AlbrechtGreatPlains1993, ArcherLonsdale2003}. North Dakota's brief reversal of this trend during the Bakken oil boom (2010--2014) demonstrated how economic shocks can temporarily reshape demographic patterns, yet the underlying structural forces favoring out-migration have proven resilient \citep{EnergyBoomDemographics2015}. Against this backdrop, international migration---particularly refugee resettlement---has assumed an outsized role in maintaining population levels and meeting labor force demands in communities throughout the state.

North Dakota's migration process is frequently mediated by administrative and humanitarian mechanisms as well as market dynamics. Refugee resettlement programs and related humanitarian channels have placed cohorts from specific conflict-affected origins into a small set of communities, producing striking origin concentrations and a migration process shaped by federal decisions, annual admissions ceilings, and local placement capacity \citep{Bansak2018}. In this setting, demography is partly policy.

Forecasting under these conditions is not just a technical exercise; it is an identification and measurement problem. Migration to peripheral states often arrives in ``lumps''---cohorts, program ramp-ups, and employer-specific recruitment---rather than as smooth flows. Federal policy can change abruptly, while local reception infrastructure can be thin and therefore fragile. Meanwhile, the empirical signals available to analysts are noisy: survey-based estimates for small populations carry wide margins of error, administrative series are collected on different time bases, and the state-level net migration target used in population accounting is itself a modeled construct. In this law-of-small-numbers regime, year-to-year deviations are not simply errors to be smoothed; they are frequently the substance of the demographic process.

% ------------------------------------------------------------------------------
% Research Gap
% ------------------------------------------------------------------------------

Despite the demographic significance of international migration to small states and rural regions, the empirical literature on state-level migration forecasting remains surprisingly thin. The dominant paradigm in migration forecasting focuses on national-level flows, employing gravity models and time series methods calibrated to large-sample settings \citep{MasseyEtAl1993, Mayda2010}. When subnational analysis appears, it typically examines major receiving states---California, Texas, Florida, New York---where sample sizes support conventional inferential approaches \citep{PortesRumbaut2006}. Small states, characterized by volatile flows, limited historical observations, and heightened sensitivity to policy shocks, have largely escaped systematic empirical scrutiny.

This lacuna reflects genuine methodological challenges. International migration to North Dakota exhibits a coefficient of variation of 82.5\%, reflecting annual flows that have ranged from 30 to over 5,000 persons within a fifteen-year span. Time series of this brevity strain the asymptotic foundations of standard econometric procedures; unit root tests lose power, ARIMA model selection becomes unreliable, and long-horizon forecasts carry prediction intervals so wide as to approach uninformativeness \citep{Cerqueira2019, HyndmanAthanasopoulos2021}. The challenge, then, is not merely to apply established methods but to develop an analytical framework suited to the small-sample, high-volatility setting that characterizes migration to peripheral regions. Such a framework must integrate multiple methodological traditions, triangulating findings across approaches while honestly characterizing the uncertainty inherent in projection.

This article advances a ``peripheral demography'' approach to international migration forecasting: combine multiple methods and data sources, keep the forecasting target explicit, and treat uncertainty as a first-class result. We define the estimand as \ND's annual net international migration component from the Census Bureau Population Estimates Program (\PEP), and use administrative and survey sources---Refugee Processing Center arrivals, DHS \LPR{} admissions, and \ACS{} foreign-born stocks---as auxiliary inputs for decomposition, mechanism checks, and predictor construction.

% ------------------------------------------------------------------------------
% Research Questions
% ------------------------------------------------------------------------------

This analysis addresses four interconnected research questions regarding international migration to North Dakota:

\begin{enumerate}
    \item \textbf{Patterns and Sources}: What are the dominant patterns and geographic sources of international migration to North Dakota, and how has the composition of this migration stream evolved over time?

	    \item \textbf{Time Series Properties}: What statistical properties characterize the international migration time series---specifically, what is the order of integration, are structural breaks present, and what degree of persistence or mean reversion do the data exhibit?

	    \item \textbf{Policy Sensitivity}: How are major federal shocks (the 2017 Travel Ban and the 2020 COVID-19 disruption) reflected in North Dakota's migration, and what are the bounds on causal interpretation in a small-sample setting?

	    \item \textbf{Future Scenarios}: Given observed volatility and policy sensitivity, what is the plausible range of net international migration paths through 2045, and how should demographic planners interpret forecast uncertainty?
\end{enumerate}

These questions progress from descriptive characterization through inferential analysis to applied forecasting, reflecting the logical structure of demographic inquiry. Each question demands distinct methodological approaches, and the answers collectively inform the population projection enterprise.

% ------------------------------------------------------------------------------
% Contributions
% ------------------------------------------------------------------------------

This study makes three principal contributions to the migration forecasting literature. First, it provides the first comprehensive multi-method empirical analysis of international migration to a small U.S. state. The analysis deploys nine interconnected methodological modules spanning descriptive statistics, time series analysis, panel regression, gravity models, machine learning, causal inference, and duration analysis. While this methodological breadth serves triangulation, the modules are not of equal weight. The Panel Data and Causal Inference models serve as the primary engines for understanding migration drivers and policy effects, respectively, while machine learning modules serve an auxiliary diagnostic role. This multi-method design explicitly addresses the small-sample challenge by trading depth within any single paradigm for robustness across paradigms.

Second, the analysis offers novel applications of causal inference methods to state-level migration policy evaluation. Difference-in-differences estimation, applied to refugee arrivals before and after the 2017 Travel Ban, estimates a policy-associated divergence of approximately -75\% (i.e., roughly 75\% lower arrivals) in the first full post years (conventional $p = 0.031$), a pattern that aggregate time series analysis fails to detect. Because the joint pre-treatment test rejects parallel trends over the full pre-period and inference is sensitive with only seven treated nationalities, this estimate is interpreted as descriptive policy-sensitivity evidence rather than a precise causal effect magnitude; restricted pre-period specifications are reported as robustness checks. Because the Travel Ban is a national shock, a synthetic comparator is used only as a descriptive benchmark for North Dakota's international migration rate, not a causal counterfactual. A shift-share index based on leave-one-out national shocks provides first-stage relevance linking refugee-driven inflows to state migration. These applications demonstrate that policy-sensitive inference remains feasible in small-state settings when identification assumptions are stated and bounded.

Third, the study develops a rigorous framework for uncertainty quantification in long-range migration forecasts. Monte Carlo simulation propagates parameter uncertainty through projection models, generating probability distributions rather than point estimates for future migration. The resulting 95\% prediction interval for 2045 international migration spans 3,407 to 14,806 persons---a factor of 4.3 times---honestly reflecting the structural uncertainty that characterizes this domain. This probabilistic framing enables demographic planners to assess the full range of contingencies rather than anchoring on misleadingly precise point forecasts \citep{RafteryProbabilistic2012}. The scenario analysis further distinguishes between forecast uncertainty (arising from model imprecision) and scenario uncertainty (arising from unknowable future policy regimes), providing a framework for structured demographic planning under deep uncertainty.

% ------------------------------------------------------------------------------
% Article Roadmap
% ------------------------------------------------------------------------------

The remainder of this article proceeds as follows. Section~\ref{sec:data_methods} describes the data sources---Census Bureau Population Estimates, Department of Homeland Security admission records, American Community Survey foreign-born tabulations, and refugee arrival databases---and presents the methodological framework in detail. Section~\ref{sec:results} reports findings organized by analytical module, beginning with descriptive patterns and concentration analysis, proceeding through time series diagnostics and panel regression results, and culminating in causal estimates and scenario projections. Section~\ref{sec:discussion} interprets these findings in light of migration theory and prior empirical work, addresses policy implications for refugee resettlement and demographic planning, and acknowledges the limitations inherent in small-sample analysis. Section~\ref{sec:conclusion} summarizes the contributions and identifies directions for future research.
