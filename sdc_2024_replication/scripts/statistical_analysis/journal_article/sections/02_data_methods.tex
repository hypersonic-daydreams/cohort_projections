% ==============================================================================
% DATA AND METHODS
% Forecasting International Migration to North Dakota: A Multi-Method Analysis
% ==============================================================================

\section{Data and Methods}
\label{sec:data_methods}

This section describes the data sources underlying the analysis and presents the methodological framework employed across nine analytical modules. The multi-method design reflects both the complexity of international migration dynamics and the inferential challenges posed by small-sample settings. Each methodological tradition contributes distinct insights: descriptive methods establish baseline patterns; time series techniques characterize dynamic properties; regression approaches quantify relationships; and causal inference methods identify policy effects. Together, these approaches enable triangulation across analytical frameworks, enhancing confidence in findings that emerge consistently while flagging conclusions dependent on particular modeling assumptions.

% ==============================================================================
% 2.0 ANALYTICAL PIPELINE OVERVIEW
% ==============================================================================

\subsection{Analytical Pipeline Overview}
\label{subsec:pipeline_overview}

Figure~\ref{fig:analysis_pipeline} summarizes the nine-module pipeline and the flow of information into the scenario engine. Modules 1--6 provide descriptive diagnostics, baseline time-series structure, and predictive covariates. Module 7 offers causal evidence of policy sensitivity, which informs the narrative interpretation of scenarios but does not replace the forecast target. Module 8 translates refugee wave persistence into a forecasting adjustment, and Module 9 integrates all inputs into scenario paths and Monte Carlo uncertainty.

\begin{figure}[htbp]
\centering
\includegraphics[width=\textwidth]{figures/analysis_pipeline.pdf}
\caption{Analytical pipeline linking data sources, module groups, and scenario outputs. Solid arrows denote the forecasting flow; dashed arrows indicate policy-evidence inputs.}
\label{fig:analysis_pipeline}
\end{figure}

\begin{table}[htbp]
\centering
\caption{Module classification and role in the forecasting pipeline}
\label{tab:module_roles}
\small
\begin{tabular}{@{}llc@{}}
\toprule
\textbf{Modules} & \textbf{Primary focus} & \textbf{Role} \\
\midrule
1.1--1.2 & Descriptive statistics and concentration & Forecasting context \\
2.1--2.2 & Time-series diagnostics and baselines & Forecasting context \\
3.1--4 & Panel and robust regression & Forecasting inputs \\
5 & Gravity and diaspora associations & Forecasting inputs \\
6 & Machine learning predictors & Forecasting inputs \\
7 & Policy evaluation (DiD, ITS) & Causal evidence \\
8 & Wave duration modeling & Forecasting input \\
9 & Scenario modeling and Monte Carlo & Forecasting output \\
\bottomrule
\end{tabular}
\begin{tablenotes}
\small
\item \textit{Notes}: Forecasting-role modules feed scenario construction; causal evidence informs policy sensitivity but does not redefine the estimand.
\end{tablenotes}
\end{table}

% ==============================================================================
% 2.1 DATA SOURCES
% ==============================================================================

\subsection{Data Sources}
\label{subsec:data_sources}

The analysis draws on four primary data sources, each capturing distinct dimensions of international migration to North Dakota. We define a single forecasting estimand based on the Census Bureau Population Estimates Program (\PEP) net international migration component for North Dakota, and treat all other sources as auxiliary inputs for decomposition, prediction, and validation. Table~\ref{tab:data_sources} maps each source to its measurement properties (flow versus stock; net versus gross; time basis) and its role relative to the forecast target.

\subsubsection{Estimand \& Measurement}

\textbf{Forecast target (estimand).} Let \(Y_t\) denote \ND's annual \emph{net international migration} in year \(t\), measured as the \PEP ``international migration'' component of change for North Dakota (persons; net).\footnote{\PEP ``year \(t\)'' corresponds to the annual component-of-change interval used in producing July 1 population estimates (a demographic year rather than a strict January--December calendar year). We refer to it as ``year \(t\)'' for readability and explicitly label fiscal-year sources as FY.} We treat \(Y_t\) as the primary dependent variable for forecasting and scenario projections in this paper. Unless otherwise noted, the generic notation \(y_t\) used in subsequent module equations denotes the observed \PEP net international migration series. Our forecasting goal is to characterize the predictive distribution of \(Y_{t+h}\) for horizons \(h \in \{1,\ldots,H\}\) (2025--2045), conditional on information available through year \(t\).

\textbf{Auxiliary data sources.} DHS lawful permanent resident (\LPR) admissions, Refugee Processing Center (RPC) refugee arrivals, and \ACS foreign-born stock estimates are not alternative outcomes; they are used to (i) characterize and partially decompose international migration, (ii) construct predictors (e.g., diaspora measures and origin-composition features), and (iii) triangulate mechanisms and scenario sensitivity.

\textbf{Time base and harmonization.} \PEP estimates are reported on an estimate-year basis, while DHS and RPC series are reported in fiscal years and \ACS estimates in survey years. We preserve native time bases and label them explicitly; when fiscal-year series are used as predictors for \PEP-year outcomes, we align them using an overlap-weighted crosswalk \(X^{\text{PEP}}_{t} \approx 0.75\,X^{\text{FY}}_{t} + 0.25\,X^{\text{FY}}_{t-1}\), and we report sensitivity to one-year shifts where timing is ambiguous.

\begin{table}[htbp]
\centering
\caption{Data Source Mapping Relative to the Forecast Target \(Y_t\)}
\label{tab:data_sources}
\begingroup
\small
\setstretch{1}
\renewcommand{\arraystretch}{0.87}
\begin{tabularx}{\textwidth}{@{}>{\raggedright\arraybackslash}p{2.6cm} >{\raggedright\arraybackslash}X >{\raggedright\arraybackslash}p{1.3cm} >{\raggedright\arraybackslash}p{1.3cm} >{\raggedright\arraybackslash}p{2.4cm} >{\raggedright\arraybackslash}X@{}}
\toprule
\textbf{Source} & \textbf{What it measures} & \textbf{\shortstack{Flow\\Stock}} & \textbf{\shortstack{Net\\Gross}} & \textbf{\shortstack{Time basis\\(coverage)}} & \textbf{Role relative to estimand and key caveats} \\
\midrule
Census \PEP{} (components of change) & State net international migration component of population change (persons) & Flow & Net & \PEP{} estimate year (2010--2024) & \textbf{Primary forecast target \(Y_t\).} Model-based estimates; composite of multiple international channels; subject to revision; 2020 reflects both true disruption and potential measurement artifacts. \\
RPC refugee arrivals & Refugee (and related humanitarian) arrivals by nationality and state of initial placement & Flow & Gross & Fiscal year (FY; 2002--2020) & Mechanism and scenario input for the refugee component; supports policy responsiveness analysis. Initial placement $\neq$ final residence (secondary migration); FY timing differs from \PEP; many zeros at state$\times$nationality. \\
\DHS\ \LPR admissions & Lawful permanent resident admissions by origin (state of intended residence) & Flow & Gross & Fiscal year (FY; 2023) & Predictor and composition diagnostics (origin structure; gravity-style features). Intended residence may differ from actual; single cross-section; excludes other statuses; not net of outflows. \\
\ACS{} foreign-born stock & Foreign-born population stock by place of birth (survey estimate; MOE) & Stock & --- & Survey year (5-year est.; 2009--2023) & Diaspora predictors and validation for origin concentration. Sampling error (MOE) is nontrivial for small cells; stock conflates past flows, retention, and internal migration; measurement error can attenuate relationships. \\
\bottomrule
\end{tabularx}
\endgroup
\end{table}

These sources are related but not interchangeable: \PEP international migration is a \emph{net} component of population accounting, DHS and RPC series are \emph{gross inflows} on a fiscal-year basis, and \ACS series are \emph{stocks} subject to sampling error. Throughout the paper we label flow/stock status and time base (FY vs.\ \PEP-year vs.\ survey year) in text, tables, and figure captions to prevent target drift.

\subsubsection{Census Bureau Population Estimates Program}

The U.S. Census Bureau's Population Estimates Program (PEP) provides annual estimates of population and components of change for states and counties. This analysis employs a spliced time series combining Vintage 2020 estimates (covering 2010--2019) and Vintage 2024 estimates (covering 2020--2024). This approach bridges the two distinct post-censal estimation regimes to analyze the full 2010--2024 period. The series decomposes annual population change into births, deaths, domestic migration, and international migration. The international migration component---designated \texttt{INTERNATIONALMIG} in Census Bureau nomenclature---comprises the net flow of foreign-born individuals crossing U.S. borders plus the net movement of native-born citizens and military personnel abroad. For subnational units, the Census Bureau allocates national international migration estimates to states using American Community Survey data on recent movers and foreign-born population distributions.

The PEP data provide the primary dependent variable for time series analysis: annual net international migration to North Dakota. Key advantages include consistent methodology across years and comprehensive geographic coverage. Limitations include the composite nature of the international migration variable, which aggregates heterogeneous streams (lawful permanent residents, refugees, temporary workers, and unauthorized migrants) into a single measure, and the allocation methodology, which may introduce smoothing artifacts at the state level.

\subsubsection{Department of Homeland Security Lawful Permanent Resident Data}

The Department of Homeland Security (DHS) publishes annual statistics on lawful permanent resident (LPR) admissions by state of intended residence and country of birth. This analysis employs fiscal year 2023 data, the most recent available at the time of analysis. The LPR data provide granular information on the country-of-origin composition of legal immigration flows, enabling gravity model estimation and concentration analysis.

The principal advantage of DHS data lies in their administrative provenance: unlike survey-based measures, LPR counts derive from visa issuance records and represent actual admissions rather than estimates. Limitations include restriction to a single cross-section (precluding panel analysis within this data source), exclusion of non-LPR categories (refugees, asylees, temporary workers), and the distinction between state of intended residence at admission and actual destination after secondary migration.

\subsubsection{American Community Survey Foreign-Born Population}

The American Community Survey (ACS) provides annual estimates of foreign-born population by place of birth for states and metropolitan areas. This analysis employs five-year estimates from 2009--2023, using the detailed country-of-birth tabulations (Table B05006). The foreign-born stock data enable analysis of diaspora effects on new migration, origin-specific growth trajectories, and Location Quotient calculations comparing North Dakota's immigrant composition to national patterns.

ACS data carry sampling uncertainty quantified through published margins of error. For small populations---such as foreign-born from specific countries residing in North Dakota---coefficients of variation can exceed 30\%, requiring cautious interpretation. The analysis addresses this limitation by focusing on origin groups with sufficient sample sizes and by aggregating to regional categories where country-level estimates prove unreliable.

\subsubsection{Refugee Processing Center Arrival Data}

The Refugee Processing Center (RPC), operated by the Department of State, maintains comprehensive records of refugee arrivals by state of initial resettlement and nationality. This analysis employs arrival data spanning fiscal years 2002--2020, providing the longest time series among the data sources and enabling duration analysis of migration ``waves'' from specific origin countries. The refugee data prove particularly valuable for North Dakota given the state's significant refugee population, which drives much of the geographic concentration observed in other data sources.

Refugee arrival data offer precise counts (not estimates) of a well-defined population admitted through federal resettlement programs. Limitations include truncation at fiscal year 2020 (precluding analysis of pandemic recovery) and restriction to initial placement, which may differ from eventual settlement after secondary migration.

% ==============================================================================
% 2.1.5 DATA COMPARABILITY ACROSS CENSUS VINTAGES
% ==============================================================================

\subsection{Data Comparability Across Census Vintages}
\label{subsec:data_comparability}

The extended time series analysis spanning 2000--2024 requires careful attention to data comparability, as the series is constructed from three distinct Census Bureau \PEP{} vintages with different underlying estimation methodologies. This subsection documents the vintage structure, explains the methodological transitions, and presents empirical evidence on whether observed patterns reflect genuine migration dynamics or measurement artifacts.

\subsubsection{PEP Vintage System and Methodology Transitions}

The Census Bureau releases annual ``vintages'' of population estimates, where each vintage supersedes previous estimates and incorporates methodological refinements anchored to decennial census benchmarks. The extended series employed in this analysis bridges three vintages:

\begin{table}[htbp]
\centering
\caption{Census Bureau PEP Vintages and Methodology Characteristics}
\label{tab:vintage_methodology}
\small
\begin{tabular}{@{}llll@{}}
\toprule
\textbf{Vintage} & \textbf{Coverage} & \textbf{Net Int'l Migration Method} & \textbf{State Allocation} \\
\midrule
Vintage 2009 & 2000--2009 & Residual from decennial census & Census + INS admin data \\
Vintage 2020 & 2010--2019 & ROYA via ACS foreign-born flows & 3-year pooled ACS \\
Vintage 2024 & 2020--2024 & ROYA + DHS humanitarian adjustment & ACS + DHS records \\
\bottomrule
\end{tabular}
\begin{tablenotes}
\small
\item \textit{Notes}: ROYA = Residual of Year of Arrival method. ACS = American Community Survey. DHS = Department of Homeland Security.
\end{tablenotes}
\end{table}

The Census Bureau explicitly cautions against combining data across vintages: ``Data from separate vintages should not be combined. Due to periodic methodological updates\ldots year-to-year comparisons in the estimates should only be done within the same vintage.'' Despite this guidance, the statistical analysis requires longer time series to achieve adequate power for unit root tests, structural break detection, and trend estimation. We therefore adopt a \emph{hybrid approach} that preserves inferential validity while incorporating robustness information from the extended series.

\subsubsection{Analysis Strategy: Primary and Extended Windows}

The analysis employs a two-track strategy:

\begin{enumerate}
    \item \textbf{Primary inference window (2010--2024, $n=15$)}: All causal claims, point estimates, and statistical tests reported in the main text derive from data within the Vintage 2020 and Vintage 2024 estimates. This window maintains methodological consistency (both vintages use ROYA-based estimation) while spanning the periods of substantive interest.

    \item \textbf{Extended robustness window (2000--2024, $n=25$)}: Sensitivity analyses incorporate the Vintage 2009 data (2000--2009) to assess whether conclusions remain stable under the longer series. These results are reported as robustness checks rather than primary findings, and we explicitly model potential vintage effects through regime-aware specifications.
\end{enumerate}

Throughout the analysis, we refer to the concatenated 2000--2024 data as a ``spliced \PEP-vintage series'' to signal that it bridges distinct measurement regimes rather than representing a methodologically homogeneous time series.

\subsubsection{Regime-Aware Modeling and Sensitivity Analysis}

To assess whether vintage transitions introduce spurious patterns, we employ piecewise trend models that allow slope and intercept shifts at vintage boundaries (2010 and 2020). The estimated piecewise slopes reveal substantial regime differences: the 2000--2009 period exhibits a modest positive trend of 39 persons per year ($\text{SE} = 17.8$), the 2010--2019 period shows an insignificant trend of 72 persons per year ($\text{SE} = 72.3$, $p = 0.32$), and the 2020--2024 period displays a steep positive trend of 1,401 persons per year ($\text{SE} = 84.7$). A formal test of slope equality across regimes strongly rejects the null hypothesis ($F_{2,19} = 124.5$, $p < 0.001$), confirming that the extended series exhibits distinct temporal dynamics across vintage boundaries.

Sensitivity analysis across six specifications reveals that key findings are moderately robust to sample definition and vintage treatment. Table~\ref{tab:sensitivity_summary} presents results across alternative specifications:

\begin{table}[htbp]
\centering
\caption{Sensitivity of Trend Estimates Across Specifications}
\label{tab:sensitivity_summary}
\small
\begin{tabular}{@{}lcccc@{}}
\toprule
\textbf{Specification} & \textbf{$n$} & \textbf{Slope} & \textbf{SE (HAC)} & \textbf{$p$-value} \\
\midrule
Primary window (2010--2024) & 15 & 183.4 & 93.7 & 0.050 \\
Extended + vintage dummies & 25 & 132.7 & 55.1 & 0.016 \\
Excluding COVID year (2020) & 24 & 98.3 & 37.4 & 0.009 \\
Post-2010 excluding COVID & 14 & 211.0 & 78.7 & 0.007 \\
Vintage 2009 only & 10 & 39.2 & 17.8 & 0.028 \\
Vintage 2020 only & 10 & 72.4 & 72.3 & 0.316 \\
\bottomrule
\end{tabular}
\begin{tablenotes}
\small
\item \textit{Notes}: All specifications use HAC (Newey-West) standard errors with 2 lags. Slope coefficients represent annual change in net international migration (persons). The primary window specification serves as the baseline for causal inference.
\end{tablenotes}
\end{table}

The trend sign is consistent across all specifications, though magnitude varies substantially (39--211 persons per year). Notably, trend significance depends on specification: the Vintage 2020--only window (2010--2019) shows an insignificant trend ($p = 0.32$), while specifications that include COVID-era data or vintage controls achieve significance. This pattern suggests that the 2021--2024 recovery period substantially influences trend estimates.

The COVID-19 pandemic introduces an additional complication: the 2020 observation (30 persons) represents a 95\% decline from 2019 and confounds vintage methodology changes with genuine pandemic disruption. Models including a COVID intervention dummy estimate the pandemic effect at $-2{,}399$ persons ($\text{SE} = 511$, $p < 0.001$), with AIC preferring the COVID-adjusted specification over the baseline trend model.

\subsubsection{Multi-State Placebo Analysis}

A natural question arises: does North Dakota's observed post-2020 surge represent a genuine migration phenomenon (potentially driven by oil industry recovery or regional economic factors) or a methodological artifact common to all states under the Vintage 2024 estimation procedures? To address this concern, we conducted a multi-state placebo analysis comparing North Dakota's vintage transition patterns against all 51 states (including DC).

The analysis computes each state's ``shift magnitude''---the change in average international migration between the 2010--2019 period (Vintage 2020) and the 2021--2024 period (Vintage 2024, excluding the COVID year 2020). If North Dakota's pattern were driven by state-specific real factors (such as the Bakken oil recovery), we would expect North Dakota to rank among the top states in shift magnitude; if the pattern reflects common methodology artifacts, North Dakota should appear unremarkable relative to other states.

Results indicate that North Dakota ranks 43rd among 51 states in shift magnitude (84th percentile from the bottom), with an increase of 1,906 persons compared to a national mean of 38,660. This ranking places North Dakota in the lower quartile of post-2020 shifts, suggesting that its recovery pattern is not exceptional relative to national trends. Among the ten major oil-producing states, North Dakota ranks 8th (ahead of only Alaska and Wyoming), and statistical tests reveal no significant difference between oil-producing and non-oil states in vintage transition patterns ($t = -0.37$, $p = 0.71$; Mann-Whitney $p = 0.43$).

These findings provide \emph{weak support} for the hypothesis that North Dakota's post-2020 migration surge reflects genuine oil-driven dynamics rather than methodology artifacts. The more parsimonious interpretation is that North Dakota's pattern is consistent with nationwide post-COVID migration recovery captured under improved Vintage 2024 estimation procedures (including enhanced DHS administrative records integration). Substantive conclusions should therefore be interpreted cautiously, with explicit acknowledgment that methodology changes may account for some portion of observed level shifts.

\subsubsection{Implications for Interpretation}

The vintage comparability analysis yields three key implications for interpreting subsequent results:

\begin{enumerate}
    \item \textbf{Level comparisons across vintage boundaries should be avoided.} The 2009--2010 and 2019--2020 transitions involve both real events (financial crisis, COVID pandemic) and methodology changes, making it impossible to attribute level shifts to either source definitively.

    \item \textbf{Within-vintage trends are more reliable.} Trend estimates within the 2010--2019 period (insignificant) and within the 2020--2024 period (strongly positive) reflect patterns under consistent methodology, though the latter is heavily influenced by COVID recovery dynamics.

    \item \textbf{Cross-vintage robustness checks inform but do not override primary conclusions.} When extended-series results qualitatively agree with primary-window results, confidence increases; when they diverge, the primary-window result takes precedence for causal interpretation.
\end{enumerate}

% ==============================================================================
% 2.2 DESCRIPTIVE AND CONCENTRATION METHODS
% ==============================================================================

\subsection{Descriptive and Concentration Methods}
\label{subsec:descriptive_methods}

Descriptive analysis establishes baseline patterns in international migration flows and characterizes their distributional properties. Summary statistics---means, standard deviations, coefficients of variation, skewness, and kurtosis---quantify central tendency and dispersion. The coefficient of variation (CV), defined as the ratio of standard deviation to mean, provides a scale-invariant measure of volatility suitable for comparing series of different magnitudes.

To decompose observed migration flows into trend and cyclical components, the analysis employs the Hodrick-Prescott (HP) filter \citep{HodrickPrescott1997}. The HP filter solves the optimization problem:
\begin{equation}
\label{eq:hp_filter}
\min_{\{g_t\}} \left\{ \sum_{t=1}^{T} (y_t - g_t)^2 + \lambda \sum_{t=2}^{T-1} [(g_{t+1} - g_t) - (g_t - g_{t-1})]^2 \right\}
\end{equation}
where $y_t$ denotes the observed series, $g_t$ the trend component, and $\lambda$ the smoothing parameter controlling the penalty on trend acceleration. Following \citet{RavnUhlig2002}, the analysis sets $\lambda = 6.25$ for annual data, a value calibrated to achieve comparable smoothing across observation frequencies.

Geographic concentration analysis employs two complementary measures. The Herfindahl-Hirschman Index (HHI) quantifies concentration across origin countries:
\begin{equation}
\label{eq:hhi}
\text{HHI} = \sum_{i=1}^{N} s_i^2 \times 10{,}000
\end{equation}
where $s_i$ denotes the share of migration from origin country $i$. HHI values below 1,500 indicate unconcentrated distributions; values between 1,500 and 2,500 indicate moderate concentration; values above 2,500 indicate high concentration. Location Quotients (LQ) measure the degree to which North Dakota's immigrant composition deviates from national patterns:
\begin{equation}
\label{eq:lq}
\text{LQ}_{i,\text{ND}} = \frac{(\text{Foreign-born from } i \text{ in ND}) / (\text{Total foreign-born in ND})}{(\text{Foreign-born from } i \text{ in US}) / (\text{Total foreign-born in US})}
\end{equation}
Location quotients exceeding unity indicate overrepresentation of origin group $i$ in North Dakota relative to the nation.

% ==============================================================================
% 2.3 TIME SERIES METHODS
% ==============================================================================

\subsection{Time Series Methods}
\label{subsec:time_series_methods}

Time series analysis characterizes the stochastic properties of international migration flows and provides a foundation for forecasting. The analysis proceeds through four stages: unit root testing, ARIMA model specification, structural break detection, and vector autoregression for multivariate dynamics.

\subsubsection{Unit Root Tests}

Stationarity assessment employs ADF and PP tests alongside KPSS (null: stationarity) to triangulate integration properties in a very short series. The Augmented Dickey-Fuller (ADF) test \citep{DickeyFuller1979} estimates the regression:
\begin{equation}
\label{eq:adf}
\Delta y_t = \alpha + \beta t + \gamma y_{t-1} + \sum_{j=1}^{p} \delta_j \Delta y_{t-j} + \varepsilon_t
\end{equation}
and tests $H_0: \gamma = 0$ (unit root) against $H_1: \gamma < 0$ (stationarity). Lag length $p$ is selected by the Akaike Information Criterion (AIC). The Phillips-Perron (PP) test \citep{PhillipsPerron1988} provides a nonparametric alternative that corrects for serial correlation without augmenting the regression. KPSS tests are run with a constant-only specification (level-stationarity). Break-robust Zivot-Andrews tests (single endogenous break) are reported as sensitivity checks. Where tests yield conflicting conclusions, the analysis reports both and interprets results cautiously.

\subsubsection{ARIMA Model Selection}

For series exhibiting unit roots, the analysis fits autoregressive integrated moving average (ARIMA) models of order $(p, d, q)$, where $d$ denotes the differencing order required for stationarity \citep{BoxJenkins1970}. Model selection employs the AIC criterion:
\begin{equation}
\label{eq:aic}
\text{AIC} = -2 \ln(\hat{L}) + 2k
\end{equation}
where $\hat{L}$ denotes the maximized likelihood and $k$ the number of parameters. Diagnostic checking employs the Ljung-Box portmanteau test for residual autocorrelation and the Shapiro-Wilk test for residual normality.

\subsubsection{Structural Break Tests}

Structural break analysis employs three complementary approaches. The Chow test \citep{Chow1960} evaluates whether regression parameters differ across known candidate break points by comparing residual sums of squares from restricted (single-regime) and unrestricted (regime-specific) models. The CUSUM test \citep{BrownDurbinEvans1975} detects parameter instability through cumulative sums of recursive residuals; departures from the zero line indicate structural change. For endogenous break detection, the analysis employs the \citet{BaiPerron1998} procedure, which identifies multiple break points by minimizing the sum of squared residuals across all possible segmentations:
\begin{equation}
\label{eq:bai_perron}
(\hat{T}_1, \ldots, \hat{T}_m) = \arg\min_{T_1, \ldots, T_m} \sum_{j=0}^{m} \sum_{t=T_j+1}^{T_{j+1}} (y_t - \bar{y}_j)^2
\end{equation}
subject to minimum segment length constraints. Given the short time series available (n = 15), the analysis sets the minimum segment length to 2 years, acknowledging the resulting power limitations.

\subsubsection{Vector Autoregression}

To examine dynamic interdependencies between North Dakota and national migration flows, the analysis estimates a vector autoregression (VAR) of order $p$:
\begin{equation}
\label{eq:var}
\mathbf{y}_t = \mathbf{c} + \sum_{i=1}^{p} \mathbf{A}_i \mathbf{y}_{t-i} + \boldsymbol{\varepsilon}_t
\end{equation}
where $\mathbf{y}_t = (y_t^{\text{ND}}, y_t^{\text{US}})'$ contains North Dakota and national international migration, $\mathbf{A}_i$ are coefficient matrices, and $\boldsymbol{\varepsilon}_t$ is a vector white noise process. Lag order selection employs AIC, and Granger causality tests evaluate directional predictive relationships. Forecast error variance decomposition quantifies the contribution of each variable to forecast uncertainty at various horizons.

% ==============================================================================
% 2.4 PANEL DATA METHODS
% ==============================================================================

\subsection{Panel Data Methods}
\label{subsec:panel_methods}

Panel data analysis exploits variation across states and time to estimate relationships unidentifiable from single-state time series. The analysis employs state-year observations spanning 51 states (including the District of Columbia) over 15 years (2010--2024), yielding 765 observations.

The fixed effects model specifies:
\begin{equation}
\label{eq:fixed_effects}
y_{it} = \alpha_i + \lambda_t + \mathbf{x}_{it}'\boldsymbol{\beta} + \varepsilon_{it}
\end{equation}
where $y_{it}$ denotes international migration to state $i$ in year $t$, $\alpha_i$ represents state-specific intercepts absorbing time-invariant heterogeneity, $\lambda_t$ captures common year effects, and $\mathbf{x}_{it}$ contains time-varying covariates. The random effects model treats $\alpha_i$ as random draws from a population distribution rather than fixed parameters to be estimated.

Model selection between fixed and random effects employs the Hausman test \citep{Hausman1978}, which evaluates whether random effects estimates exhibit bias due to correlation between $\alpha_i$ and regressors. Under the null hypothesis of no correlation, random effects is efficient; under the alternative, fixed effects is consistent but random effects is biased. The test statistic follows a chi-squared distribution under the null:
\begin{equation}
\label{eq:hausman}
H = (\hat{\boldsymbol{\beta}}_{\text{FE}} - \hat{\boldsymbol{\beta}}_{\text{RE}})' [\text{Var}(\hat{\boldsymbol{\beta}}_{\text{FE}}) - \text{Var}(\hat{\boldsymbol{\beta}}_{\text{RE}})]^{-1} (\hat{\boldsymbol{\beta}}_{\text{FE}} - \hat{\boldsymbol{\beta}}_{\text{RE}}) \sim \chi^2_k
\end{equation}

Standard errors are clustered at the state level to account for within-state serial correlation.

% ==============================================================================
% 2.5 GRAVITY AND NETWORK MODELS
% ==============================================================================

\subsection{Cross-Sectional Allocation Models}
\label{subsec:gravity_methods}

Cross-sectional allocation models quantify how LPR admissions are distributed across state--origin pairs as a function of diaspora stocks and destination characteristics. Although inspired by the gravity framework \citep{Tinbergen1962, AndersonVanWincoop2003}, our specification differs in two respects: (1) it uses a single FY2023 cross-section rather than bilateral panel data, and (2) it omits origin--destination distance because our destination units are U.S.\ states receiving flows from the same origin countries---variation in distance to a given origin is minimal and dominated by within-U.S.\ allocation patterns rather than international migration costs.\footnote{Traditional gravity models estimate distance effects using cross-country bilateral flows where distance proxies transaction costs. In our state-level allocation context, the relevant ``distance'' would be origin--state distance, which is nearly constant across states for a given origin (e.g., distance from Nepal to North Dakota versus to Minnesota differs by less than 2\%). We therefore focus on diaspora networks and destination scale as the primary allocation mechanisms.}

The analysis estimates specifications using the Poisson pseudo-maximum likelihood (PPML) estimator advocated by \citet{SantosSilvaTenreyro2006}, which addresses two limitations of log-linear OLS: heteroskedasticity-induced bias and the inability to accommodate zero flows.

The estimated specification takes the form:
\begin{equation}
\label{eq:gravity}
E[M_{od}] = \exp(\beta_0 + \beta_1 \ln \text{Stock}_{od} + \beta_2 \ln \text{OriginTotal}_o + \beta_3 \ln \text{DestTotal}_d + \boldsymbol{\gamma}'\mathbf{Z}_{od})
\end{equation}
where $M_{od}$ denotes LPR admissions from origin country $o$ to destination state $d$ (FY2023), $\text{Stock}_{od}$ is the existing diaspora (foreign-born from $o$ residing in $d$), $\text{OriginTotal}_o$ is the national foreign-born population from origin $o$, $\text{DestTotal}_d$ is the total foreign-born population in destination state $d$, and $\mathbf{Z}_{od}$ contains additional bilateral controls. The coefficient $\beta_1$ on diaspora stock captures a cross-sectional diaspora--flow association useful for prediction.

Diaspora elasticity estimation acknowledges endogeneity concerns: diaspora stocks and migration flows are jointly determined, and historical flows mechanically generate stocks. The analysis therefore frames coefficients as predictive associations rather than causal effects; robustness checks consider lagged stock measures and historical settlement patterns \citep{Card2001}.

% ==============================================================================
% 2.6 MACHINE LEARNING METHODS
% ==============================================================================

\subsection{Machine Learning Methods}
\label{subsec:ml_methods}

Machine learning methods complement traditional regression by providing flexible functional forms and systematic variable selection. The analysis employs three techniques: Elastic Net regularization, Random Forest regression, and K-means clustering.

Elastic Net \citep{ZouHastie2005} combines $L_1$ (Lasso) and $L_2$ (Ridge) penalties, solving:
\begin{equation}
\label{eq:elastic_net}
\hat{\boldsymbol{\beta}} = \arg\min_{\boldsymbol{\beta}} \left\{ \sum_{i=1}^{n} (y_i - \mathbf{x}_i'\boldsymbol{\beta})^2 + \lambda \left[ \alpha \|\boldsymbol{\beta}\|_1 + (1-\alpha) \|\boldsymbol{\beta}\|_2^2 \right] \right\}
\end{equation}
where $\lambda$ controls overall regularization strength and $\alpha \in [0,1]$ balances sparsity-inducing $L_1$ penalty against coefficient-shrinking $L_2$ penalty. Cross-validation selects $(\lambda, \alpha)$ values. Coefficients shrunk exactly to zero identify variables excluded from the predictive model.

Random Forest \citep{Breiman2001} aggregates predictions from an ensemble of decision trees, each trained on bootstrap samples with random feature subsets. Feature importance is measured by permutation importance: the decrease in out-of-bag prediction accuracy when feature values are randomly shuffled.

K-means clustering partitions states into groups exhibiting similar migration profiles. The algorithm minimizes within-cluster sum of squared distances to cluster centroids:
\begin{equation}
\label{eq:kmeans}
\arg\min_{\mathcal{C}} \sum_{k=1}^{K} \sum_{i \in C_k} \|\mathbf{x}_i - \boldsymbol{\mu}_k\|^2
\end{equation}
where $C_k$ denotes cluster $k$ and $\boldsymbol{\mu}_k$ its centroid. Optimal cluster count $K$ is selected by the silhouette criterion.

% ==============================================================================
% 2.7 CAUSAL INFERENCE METHODS
% ==============================================================================

\subsection{Causal Inference Methods}
\label{subsec:causal_methods}

Causal inference methods estimate the effects of specific policy interventions---the 2017 Travel Ban and the 2020 COVID-19 pandemic---on international migration flows. Identification relies on nationality-level difference-in-differences and a state-level interrupted time series. Two additional components provide context: a synthetic comparator for descriptive benchmarking and a shift-share index used to assess the predictive relevance of refugee-driven shocks.

\subsubsection{Difference-in-Differences}

The Travel Ban analysis employs a difference-in-differences (DiD) design comparing refugee arrivals from affected countries (Iran, Iraq, Libya, Somalia, Sudan, Syria, Yemen) to arrivals from unaffected countries, before and after the 2017 implementation. The estimating equation specifies:
\begin{equation}
\label{eq:did}
\ln(y_{ct} + 1) = \alpha_c + \lambda_t + \delta \cdot (\text{Affected}_c \times \text{Post}_t) + \varepsilon_{ct}
\end{equation}
where $y_{ct}$ denotes arrivals from country $c$ in year $t$, $\alpha_c$ and $\lambda_t$ are country and year fixed effects, and the coefficient $\delta$ identifies the average treatment effect on the treated (ATT). The logarithmic transformation accommodates the multiplicative nature of policy effects; the addition of unity handles zero arrivals.

Identification requires the parallel trends assumption: absent the Travel Ban, affected and unaffected countries would have exhibited similar trends in arrivals. The analysis evaluates this assumption through pre-treatment trend tests and graphical inspection of pre-period trajectories. Standard errors are clustered by nationality to account for serial correlation within origin groups.

\textbf{Small-sample inference robustness.} With only seven treated nationalities, conventional cluster-robust inference may be unreliable \citep{CameronGelbachMiller2008}. We supplement clustered standard errors with two small-sample approaches. First, wild cluster bootstrap with Rademacher weights generates the null distribution by resampling cluster-level residuals under the null hypothesis of no treatment effect; bootstrap $p$-values correct for the small number of treated clusters. Second, randomization (permutation) inference implements Fisher's exact test by permuting the treatment assignment across clusters and computing the proportion of permutations yielding an ATT as extreme as observed. Both approaches provide finite-sample valid inference without relying on asymptotic cluster count assumptions.

\textbf{Restricted pre-period robustness.} When the joint pre-trend test rejects parallel trends over the full pre-period, we re-estimate the DiD restricting to years where pre-trends appear more plausibly parallel. If event study coefficients show greater divergence in early years than in years immediately preceding treatment, restricting the pre-period strengthens the parallel trends assumption at the cost of reduced sample size.

\subsubsection{Interrupted Time Series (National System Diagnostic)}

To characterize the global migration regime shift associated with the pandemic, we employ a state-level interrupted time series (ITS) model estimated on the full 50-state panel. This analysis is intended as a \emph{national system diagnostic} to quantify the average state-level shock, not as a North Dakota-specific impact estimate. The model specifies:
\begin{equation}
\label{eq:its}
y_{st} = \alpha_s + \beta_1 t + \beta_2 \text{Post}_{2020,t} + \beta_3 (t - 2020)\text{Post}_{2020,t} + \varepsilon_{st}
\end{equation}
where $y_{st}$ is net international migration for state $s$ in year $t$. The coefficient $\beta_2$ captures the immediate average COVID-19 level shift, and $\beta_3$ captures the change in trend after 2020. Standard errors are clustered by state. By estimating these parameters on the national panel, we establish the systemic volatility context within which North Dakota's specific recovery operates.

\subsubsection{Synthetic Comparator (Descriptive Benchmark)}

Because the Travel Ban is a national shock, donor states are not untreated. We therefore use a synthetic comparator \citep{AbadieGardeazabal2003, AbadieDiamondHainmueller2010} only as a descriptive benchmark. The comparator trajectory is constructed by weighting donor states to match North Dakota's pre-2017 international migration rate:
\begin{equation}
\label{eq:synth}
\hat{y}_{1t}^{S} = \sum_{j=2}^{J+1} w_j^* y_{jt}
\end{equation}
with weights $w_j^* \geq 0$, $\sum_j w_j^* = 1$ chosen to minimize pre-treatment discrepancy. Post-2017 gaps $y_{1t} - \hat{y}_{1t}^{S}$ are reported descriptively rather than interpreted as causal effects.

\subsubsection{Shift-Share (Bartik) Index}

To assess the predictive relevance of refugee-driven shocks for state-level migration, the analysis constructs a shift-share (Bartik) index \citep{Bartik1991, GoldsmithPinkhamSorkinSwift2020}. The index combines national-level ``shifts'' in immigration by origin country with state-level ``shares'' reflecting historical settlement patterns:
\begin{equation}
\label{eq:bartik}
B_{dt} = \sum_{o} \omega_{od,t_0} \cdot g_{o,t}^{\text{US},-d}
\end{equation}
where $\omega_{od,t_0}$ is origin $o$'s share of state $d$'s baseline arrivals and $g_{o,t}^{\text{US},-d}$ is the leave-one-out national change in arrivals from origin $o$ (excluding state $d$). Because no second-stage causal model is estimated, the results are interpreted as first-stage relevance: the strength of the relationship between predicted refugee inflows and state-level international migration. Standard errors are clustered by state; first-stage F-statistics assess instrument strength, with values exceeding 10 indicating adequate relevance.

% ==============================================================================
% 2.8 DURATION ANALYSIS
% ==============================================================================

\subsection{Duration Analysis}
\label{subsec:duration_methods}

Duration analysis examines the lifecycle of migration ``waves''---sustained periods of elevated arrivals from specific origin countries. A wave is defined operationally as a period during which arrivals from a nationality exceed 150\% of that nationality's baseline for two or more consecutive years. This definition balances sensitivity (detecting genuine surges) against specificity (avoiding false positives from random fluctuation).

The Kaplan-Meier estimator \citep{KaplanMeier1958} provides nonparametric estimates of the survival function $S(t) = P(T > t)$, where $T$ denotes wave duration:
\begin{equation}
\label{eq:km}
\hat{S}(t) = \prod_{t_i \leq t} \left( 1 - \frac{d_i}{n_i} \right)
\end{equation}
where $d_i$ is the number of waves ending at time $t_i$ and $n_i$ is the number at risk. The log-rank test evaluates whether survival curves differ across strata (e.g., intensity quartiles, origin regions).

The Cox proportional hazards model \citep{Cox1972} relates the hazard rate---the instantaneous probability of wave termination conditional on survival---to covariates:
\begin{equation}
\label{eq:cox}
h(t|\mathbf{x}) = h_0(t) \exp(\boldsymbol{\beta}'\mathbf{x})
\end{equation}
where $h_0(t)$ is the baseline hazard and $\exp(\beta_j)$ gives the hazard ratio associated with a one-unit increase in covariate $x_j$. Hazard ratios below unity indicate factors prolonging waves; ratios above unity indicate factors accelerating termination. The proportional hazards assumption---that hazard ratios remain constant over time---is evaluated through Schoenfeld residual tests. Model discrimination is assessed by the concordance index (C-statistic), which measures the probability that predicted hazards correctly rank pairs of observations by survival time.

% ==============================================================================
% 2.9 SCENARIO CONSTRUCTION
% ==============================================================================

\subsection{Module-to-Scenario Integration}
\label{subsec:module_scenario_integration}

Table~\ref{tab:module_outputs} summarizes how outputs from each analytical module feed into the scenario engine. This mapping clarifies the role of each module in the forecasting pipeline: some modules provide direct forecasting inputs (parameter estimates, uncertainty bounds), while others provide contextual evidence that informs scenario interpretation but does not alter the forecast mechanics.

\begin{table}[htbp]
\centering
\caption{Module Outputs and Scenario Integration}
\label{tab:module_outputs}
\small
\begin{tabular}{@{}>{\raggedright\arraybackslash}p{2.2cm} >{\raggedright\arraybackslash}p{3.5cm} >{\raggedright\arraybackslash}p{4.5cm} >{\raggedright\arraybackslash}p{2.8cm}@{}}
\toprule
\textbf{Module} & \textbf{Key Output(s)} & \textbf{Role in Scenario Engine} & \textbf{Scenarios Affected} \\
\midrule
1.1--1.2 (Descriptive) & Summary statistics; HHI; Location Quotients & Context for origin composition assumptions; validates concentration patterns & All (interpretation) \\
2.1 (ARIMA) & Baseline drift ($\hat{\beta}$); residual variance ($\hat{\sigma}^2$) & ARIMA point forecasts anchor Moderate and CBO scenarios; variance feeds MC simulation & Moderate, CBO Full \\
2.2 (VAR) & AIC model weights; FEVD proportions & Model averaging weights for ensemble; not used for standalone forecast & Model ensemble \\
3--4 (Panel/Robust) & State/year fixed effects; covariate elasticities & Validates ND position in state distribution; informs plausibility bounds & All (validation) \\
5 (Cross-sectional) & Diaspora elasticity (predictive) & Context for composition stability; not directly in forecast & None (context) \\
6 (ML) & Feature importance rankings; cluster membership & Auxiliary validation; identifies ND peer states for comparison & None (context) \\
7 (Policy) & DiD ATT bounds; ITS level/trend shifts & Informs restrictive vs.\ permissive scenario multipliers; policy sensitivity & Pre-2020 Trend \\
8 (Duration) & Cox survival probabilities; wave lifecycle shape & Wave persistence draws for MC; conditional survival by intensity/region & MC simulation \\
9 (Scenarios) & Combined trajectory paths; MC prediction intervals & Final output & All \\
\bottomrule
\end{tabular}
\begin{tablenotes}
\small
\item \textit{Notes}: ``Context'' modules provide interpretive background but do not mechanically alter scenario trajectories. ``Forecasting input'' modules contribute parameter estimates or distributional draws to the scenario engine. MC = Monte Carlo.
\end{tablenotes}
\end{table}

As Table~\ref{tab:module_outputs} indicates, the VAR model (Module 2.2) and machine learning methods (Module 6) serve primarily as auxiliary analyses. The VAR model contributes to AIC-based model averaging weights but is not used for standalone long-horizon forecasting due to the infeasibility of obtaining future national migration values. Similarly, ML methods (Elastic Net, Random Forest, K-means clustering) provide feature importance rankings and state clustering for comparative context, but do not directly parameterize the scenario trajectories. These modules support triangulation and validation rather than primary forecast generation.

\subsection{Scenario Construction}
\label{subsec:scenario_methods}

Scenario analysis translates analytical findings into forward-looking projections while explicitly characterizing uncertainty. The analysis develops four policy-indexed scenarios spanning the range of plausible future trajectories:

\begin{enumerate}
    \item \textbf{CBO Full}: Uses ARIMA point forecasts scaled by 1.1 for 2025--2029, then compounds at 8\% annually from 2030 onward. This scenario represents an explicit upper-bound stress test predicated on significant federal policy liberalization and sustained robust growth, shielding planning limits against extreme upside surprises.

    \item \textbf{Moderate}: Uses ARIMA point forecasts for 2025--2029, then applies a dampened trend (50\% of the averaged robust trend estimate) through 2045. This scenario represents the central tendency forecast.

    \item \textbf{Pre-2020 Trend}: Anchors at the 2019 level and extrapolates the 2010--2019 slope, treating COVID as a temporary deviation. This scenario provides a lower-bound estimate under restrictive policy assumptions.

    \item \textbf{Immigration Policy}: Modifies the Moderate baseline by applying a 0.65 multiplier to future flows ($ 65\% $ of baseline). This shock magnitude is derived from the conservative lower bound of the Travel Ban DiD estimate (approx. $-75\%$, treated here conservatively as $-35\%$ impact on net total flows to account for non-refugee streams).

    \item \textbf{Zero}: Assumes complete cessation of international migration, providing a floor for population projection under extreme policy scenarios.
\end{enumerate}

Monte Carlo simulation propagates uncertainty through projection models by sampling from estimated parameter distributions. The procedure draws 1,000 realizations of model parameters, generates projected trajectories for each draw, and summarizes the resulting distribution through percentiles. The 50\% prediction interval spans the 25th to 75th percentiles; the 95\% prediction interval spans the 2.5th to 97.5th percentiles. This approach quantifies forecast uncertainty arising from parameter estimation imprecision while holding the structural model fixed.

Model averaging combines forecasts from multiple specifications using AIC-derived weights:
\begin{equation}
\label{eq:model_avg}
w_m = \frac{\exp(-\Delta_m / 2)}{\sum_{j=1}^{M} \exp(-\Delta_j / 2)}
\end{equation}
where $\Delta_m = \text{AIC}_m - \text{AIC}_{\min}$ is the AIC difference between model $m$ and the best-fitting model. Weighted averaging reduces dependence on any single specification while preserving interpretability \citep{HyndmanAthanasopoulos2021}.

% ==============================================================================
% 2.10 IMPLEMENTATION
% ==============================================================================

\subsection{Implementation}
\label{subsec:implementation}

All analyses were conducted in Python 3.11. Time series analysis employed \texttt{statsmodels} for ARIMA, VAR, and unit root tests. Panel data models were estimated using \texttt{linearmodels}. Machine learning methods used \texttt{scikit-learn} for Elastic Net, Random Forest, and clustering. Survival analysis employed the \texttt{lifelines} package for Kaplan-Meier estimation and Cox regression. Structural break detection used the \texttt{ruptures} package. Replication code and data are available in the supplementary materials.
